%% Generated by Sphinx.
\def\sphinxdocclass{jupyterBook}
\documentclass[a4paper,12pt,english]{jupyterBook}
\ifdefined\pdfpxdimen
   \let\sphinxpxdimen\pdfpxdimen\else\newdimen\sphinxpxdimen
\fi \sphinxpxdimen=.75bp\relax
\ifdefined\pdfimageresolution
    \pdfimageresolution= \numexpr \dimexpr1in\relax/\sphinxpxdimen\relax
\fi
%% let collapsible pdf bookmarks panel have high depth per default
\PassOptionsToPackage{bookmarksdepth=5}{hyperref}
%% turn off hyperref patch of \index as sphinx.xdy xindy module takes care of
%% suitable \hyperpage mark-up, working around hyperref-xindy incompatibility
\PassOptionsToPackage{hyperindex=false}{hyperref}
%% memoir class requires extra handling
\makeatletter\@ifclassloaded{memoir}
{\ifdefined\memhyperindexfalse\memhyperindexfalse\fi}{}\makeatother

\PassOptionsToPackage{warn}{textcomp}

\catcode`^^^^00a0\active\protected\def^^^^00a0{\leavevmode\nobreak\ }
\usepackage{cmap}
\usepackage{fontspec}
\defaultfontfeatures[\rmfamily,\sffamily,\ttfamily]{}
\usepackage{amsmath,amssymb,amstext}
\usepackage{polyglossia}
\setmainlanguage{english}



\setmainfont{FreeSerif}[
  Extension      = .otf,
  UprightFont    = *,
  ItalicFont     = *Italic,
  BoldFont       = *Bold,
  BoldItalicFont = *BoldItalic
]
\setsansfont{FreeSans}[
  Extension      = .otf,
  UprightFont    = *,
  ItalicFont     = *Oblique,
  BoldFont       = *Bold,
  BoldItalicFont = *BoldOblique,
]
\setmonofont{FreeMono}[
  Extension      = .otf,
  UprightFont    = *,
  ItalicFont     = *Oblique,
  BoldFont       = *Bold,
  BoldItalicFont = *BoldOblique,
]



\usepackage[Bjarne]{fncychap}
\usepackage[,numfigreset=1,mathnumfig]{sphinx}

\fvset{fontsize=\small}
\usepackage{geometry}
\usepackage{nicefrac, xfrac}


% Include hyperref last.
\usepackage{hyperref}
% Fix anchor placement for figures with captions.
\usepackage{hypcap}% it must be loaded after hyperref.
% Set up styles of URL: it should be placed after hyperref.
\urlstyle{same}


\usepackage{sphinxmessages}



        % Start of preamble defined in sphinx-jupyterbook-latex %
         \usepackage[Latin,Greek]{ucharclasses}
        \usepackage{unicode-math}
        % fixing title of the toc
        \addto\captionsenglish{\renewcommand{\contentsname}{Contents}}
        \hypersetup{
            pdfencoding=auto,
            psdextra
        }
        % End of preamble defined in sphinx-jupyterbook-latex %
        

\title{Theory}
\date{Jun 09, 2023}
\release{}
\author{Zuzanna Wieczorek, Grzegorz Gruszczyński}
\newcommand{\sphinxlogo}{\vbox{}}
\renewcommand{\releasename}{}
\makeindex
\begin{document}

\pagestyle{empty}
\sphinxmaketitle
\clearpage

\pagestyle{plain}
\sphinxtableofcontents
\pagestyle{normal}
\phantomsection\label{\detokenize{chapters/description/theory::doc}}


\sphinxAtStartPar
The Vortex Lattice Method (VLM) is a numerical method used in computational fluid dynamics. VLM models a surface on aircraft as infinite vortices to estimate the lift curve slope, induced drag, and force distribution. The VLM is the extension of Prandtl’s lifting\sphinxhyphen{}line theory that is capable of computing swept and low aspect ratio wings. {[}\hyperlink{cite.chapters/bibliography:id10}{SHR+19}{]}.


\part{Introduction}
\label{\detokenize{chapters/description/theory:introduction}}
\sphinxAtStartPar
Vortex Lattice Method is build on the Potential flow theory. The viscous effects, drag and flow separation in VLM is sufficient for aproximatting ideal flow seen in nature.


\part{Potential flow theory}
\label{\detokenize{chapters/description/theory:potential-flow-theory}}
\sphinxAtStartPar
Potenttial Flow Theory treats external flows around bodies as invicid and irrotational {[}\hyperlink{cite.chapters/bibliography:id6}{Tec05}{]} .The viscous effects are limitted to a thin layer next to the body (boundary layer). Becouse of this and separation phenomena such fluid can be modelled only with small angle of attack. In irrotational flow fluid particles are not rotating.

\sphinxAtStartPar
In Potential Flow Theory, because velocity must satisfy the conservation of mass equation, the Laplace Equation is qoverning:
\begin{equation*}
\begin{split}
\nabla^{2} \phi = 0
\end{split}
\end{equation*}
\sphinxAtStartPar
where \(\phi\) is a potentinal function defined as continous function which satisfies the conservation of mass and momentum (incompressible, inviscid and irrational flow).


\part{Vortex Lattice Method}
\label{\detokenize{chapters/description/theory:vortex-lattice-method}}
\sphinxAtStartPar
Vortex Lattice Method allows the computation of induced wind. VLM is a panel method where wing or other configuration is modelled by a large number of elementary, quadrilateral panels lying either on the actual aircraft surface, or on some mean surface, or combination thereof. To satisfy boundary conditions VLM uses following elements attached to each panel:
\begin{itemize}
\item {} 
\sphinxAtStartPar
source \sphinxhyphen{} a point from which fluid issues and flows radially outward such that the continuity equation is satisfied everywhere but at the singularity that exists at the source’s center

\item {} 
\sphinxAtStartPar
sink \sphinxhyphen{} a negative source

\item {} 
\sphinxAtStartPar
doublet \sphinxhyphen{}  singularity resulting when a source or a sink of equal strength are made to approach each other such that the product of their strangths and their distance apart remains constant at a preselected finite value in the limit as a distance between them approaches zero

\item {} 
\sphinxAtStartPar
vortex \sphinxhyphen{} an element that generates a circulation, or tangential motion, around its origin

\end{itemize}

\sphinxAtStartPar
Such singularities are defined by specifying functional variation across the panel and its value is set by determinating strength parameters. Such parameters are known after solving appropriate boundary condition equations. When singularity strengts are determinated, the pressure, velocity can be computed{[}\hyperlink{cite.chapters/bibliography:id5}{JJB09}{]}.

\begin{figure}[htbp]
\centering
\capstart

\noindent\sphinxincludegraphics[scale=0.8]{{panels}.png}
\caption{Representation of an airplane flowfield by panel (or singularity) methods. Figure taken from {[}\hyperlink{cite.chapters/bibliography:id5}{JJB09}{]} (Fig. 7.22 page 350).}\label{\detokenize{chapters/description/theory:panel-method}}\end{figure}


\chapter{Vortices}
\label{\detokenize{chapters/description/theory:vortices}}
\sphinxAtStartPar
The vortex theorems for inviscid incompressible flows has been developed by German scientist Hermann von Helmholtz (1821\sphinxhyphen{}1894). The theory is based on the following assumptions {[}\hyperlink{cite.chapters/bibliography:id4}{JK01}{]}:
\begin{itemize}
\item {} 
\sphinxAtStartPar
Flow is inviscid and incompressible

\item {} 
\sphinxAtStartPar
The strength of vortex filament is constant along its length

\item {} 
\sphinxAtStartPar
A vortex filament cannot start or end in a fluid (it must form or closed path or extend to infinity)

\item {} 
\sphinxAtStartPar
The fluid that forms a vortex tube continues to form a vortex tube and the strength of the vortex tube remains constant as the tube moves about.

\end{itemize}


\section{Vortex element}
\label{\detokenize{chapters/description/theory:vortex-element}}
\sphinxAtStartPar
Vortex element in 2D is presented on figure \hyperref[\detokenize{chapters/description/theory:vortex}]{\ref{\detokenize{chapters/description/theory:vortex}}}. It is created by placing a rotating cylindrical core in two dimentional flowfield. If the cylinder has a radius \(R\) and a constant angular velocity of \(\omega_y\), then its motion results in a flow with circular streamlines {[}\hyperlink{cite.chapters/bibliography:id4}{JK01}{]}.

\begin{figure}[htbp]
\centering
\capstart

\noindent\sphinxincludegraphics[scale=0.5]{{vort}.png}
\caption{Two dimentional flowfield around a cylindrical core rotating as a rigid body. Figure taken from {[}\hyperlink{cite.chapters/bibliography:id4}{JK01}{]} (Fig. 2.11 page 34).}\label{\detokenize{chapters/description/theory:vortex}}\end{figure}

\sphinxAtStartPar
The tangential velocity induced is described by equation:
\begin{equation*}
\begin{split}
q_\theta(r) = \frac{\Gamma}{2\pi \cdot r}
\end{split}
\end{equation*}
\sphinxAtStartPar
where \(r\) is described as distance from the vortex core.

\sphinxAtStartPar
If a vortex is located in free\sphinxhyphen{}strem with uniform velocity \(u_\infty\) then the total velocity at the distance \(r\) can be writtes as \(\overrightarrow{u_\infty} + \overrightarrow{q_\theta(r)}\) {[}\hyperlink{cite.chapters/bibliography:id3}{AH13}{]}.


\section{Vortex segment}
\label{\detokenize{chapters/description/theory:vortex-segment}}
\begin{figure}[htbp]
\centering
\capstart

\noindent\sphinxincludegraphics[scale=0.5]{{vor_seg}.png}
\caption{Vortex segment in three dimentions. Figure taken from {[}\hyperlink{cite.chapters/bibliography:id4}{JK01}{]} (Fig. 2.16 page 40).}\label{\detokenize{chapters/description/theory:id10}}\end{figure}

\sphinxAtStartPar
The velocity induced by a straight vortex segement (figure \hyperref[\detokenize{chapters/description/theory:id10}]{\ref{\detokenize{chapters/description/theory:id10}}}) in three dimentions with given vortex strength \(\Gamma\) at point \(P(x,y,z)\) is qiven by equation:
\begin{equation}\label{equation:chapters/description/theory:q_finite}
\begin{split}
\overrightarrow{w_{1,2}} = \frac{\Gamma}{4\pi}\frac{\overrightarrow{r_1} \times \overrightarrow{r_2}}{|\overrightarrow{r_1} \times \overrightarrow{r_2}|^2} \left(\frac{\overrightarrow{r_1}}{||\overrightarrow{r_1}||} - \frac{\overrightarrow{r_2}}{||\overrightarrow{r_2}||}\right) \cdot\overrightarrow{r_0}=\overrightarrow{\nu}\,\Gamma
\end{split}
\end{equation}
\sphinxAtStartPar
where
\begin{equation*}
\begin{split}
\overrightarrow{r_{0}} = \overrightarrow{r_1} - \overrightarrow{r_2}
\end{split}
\end{equation*}
\sphinxAtStartPar
and
\begin{gather*}
\overrightarrow{r_{0}} = (x_2, y_2, z_2) - (x_1, y_1, z_1) \\
\overrightarrow{r_{1}} = P - (x_1, y_1, z_1)\\
\overrightarrow{r_{2}} = P - (x_2, y_2, z_2)
\end{gather*}
\sphinxAtStartPar
In case of a semi\sphinxhyphen{}infinite vortex line, the point 2 is infinitely far away thus formula reads {[}\hyperlink{cite.chapters/bibliography:id8}{PS00}{]}:
\begin{equation}\label{equation:chapters/description/theory:q_infinite}
\begin{split}
\overrightarrow{w_{1,2}} = \frac{\Gamma}{4\pi}\frac{\overrightarrow{u_\infty} \times \overrightarrow{r_1}}{||\overrightarrow{r_1}||(||\overrightarrow{r_1}|| - \overrightarrow{u_\infty} \cdot \overrightarrow{r_1})}=\overrightarrow{\nu}\,\Gamma
\end{split}
\end{equation}

\chapter{The Kutta Condition}
\label{\detokenize{chapters/description/theory:the-kutta-condition}}
\sphinxAtStartPar
The Kutta Condition states that at small angle of attack the flow leaves the sharp trailing edge of an airfoil smoothly and the velocity is finite there {[}\hyperlink{cite.chapters/bibliography:id4}{JK01}{]}. Because of this the normal component of velocity, from both sides of the airfoil, must vanish. Circulation at trailing edge can be expressed by equation:
\begin{equation*}
\begin{split}
\gamma_{T.E.}=0
\end{split}
\end{equation*}
\begin{figure}[htbp]
\centering
\capstart

\noindent\sphinxincludegraphics[scale=0.5]{{kutta}.png}
\caption{Flow near cusped trailing edge. Figure taken from {[}\hyperlink{cite.chapters/bibliography:id4}{JK01}{]} (Fig. 4.12 page 89)}\label{\detokenize{chapters/description/theory:kutta}}\end{figure}


\chapter{Lifting surface}
\label{\detokenize{chapters/description/theory:lifting-surface}}
\begin{figure}[htbp]
\centering
\capstart

\noindent\sphinxincludegraphics[height=300\sphinxpxdimen]{{siatka}.png}
\caption{Nomenclature of the lattice elements. Figure created by author.}\label{\detokenize{chapters/description/theory:siatka}}\end{figure}

\sphinxAtStartPar
The surface of the object is divided into panels (gray rectanges on figure \hyperref[\detokenize{chapters/description/theory:siatka}]{\ref{\detokenize{chapters/description/theory:siatka}}} ). The total amount of them is a product of number of panels spanwise and chordwise. Vortex element is associated to each one of them (pink rectangles).

\sphinxAtStartPar
There are two types of vortex elements:
\begin{itemize}
\item {} 
\sphinxAtStartPar
vortex ring

\item {} 
\sphinxAtStartPar
vortex horseshoe

\end{itemize}

\begin{figure}[htbp]
\centering
\capstart

\noindent\sphinxincludegraphics[height=400\sphinxpxdimen]{{ring}.png}
\caption{Nomenclature of the vortex ring. Figure created by author.}\label{\detokenize{chapters/description/theory:ring}}\end{figure}

\sphinxAtStartPar
Vortex ring (figure \hyperref[\detokenize{chapters/description/theory:ring}]{\ref{\detokenize{chapters/description/theory:ring}}}) is created by four finite segments according to equation \eqref{equation:chapters/description/theory:q_finite}. Horseshoe vortex is attached to the trailing edge of lifting surface. It consists of three finite and two semi\sphinxhyphen{}infinite segments following equation \eqref{equation:chapters/description/theory:q_finite} and \eqref{equation:chapters/description/theory:q_infinite} (see figure \hyperref[\detokenize{chapters/description/theory:horseshoe}]{\ref{\detokenize{chapters/description/theory:horseshoe}}}). By placing vortex at the quarter chord line of the the two\sphinxhyphen{}dimensional Kutta condition is satisfied along the chord.  Also, along the wing trailing edges, the trailing vortex of the last panel row must be canceled to satisfy the three\sphinxhyphen{}dimensional trailing\sphinxhyphen{}edge condition (see figure \hyperref[\detokenize{chapters/description/theory:tip-vor}]{\ref{\detokenize{chapters/description/theory:tip-vor}}}).

\begin{figure}[htbp]
\centering
\capstart

\noindent\sphinxincludegraphics[height=400\sphinxpxdimen]{{horseshoe}.png}
\caption{Nomenclature of the vortex horseshoe. Figure created by author.}\label{\detokenize{chapters/description/theory:horseshoe}}\end{figure}

\begin{figure}[htbp]
\centering
\capstart

\noindent\sphinxincludegraphics[height=200\sphinxpxdimen]{{Tip-vortices}.png}
\caption{The total amount of circulation is preserved. Figure 6 from {[}\hyperlink{cite.chapters/bibliography:id7}{Gru21}{]}.}\label{\detokenize{chapters/description/theory:tip-vor}}\end{figure}

\sphinxAtStartPar
When thin surface is angled to a free\sphinxhyphen{}stream \(\overrightarrow{u_{\infty}}\), the aerodynamic force is being generated at center of pressure. This point is located at the \(1/4\) of a panel chord and at \(1/2\) of the panel span (from panel leading edge). To fulfill no flow through the surface, the control point is defined at \(3/4\) of the chord from the panel leading edge and in the middle of the span.

\sphinxAtStartPar
Vortex vertices \(B_k\) and \(C_k\) are placed at \(1/4\) of panel chord (\(v_{k}\)), \(A_k\) and \(D_k\) at \(1/4\) of the next panel (\(v_{k+1}\)). The panel opposite corner points define two vectors \(\overrightarrow{A_k}\) and \(\overrightarrow{B_k}\), and their vector product will point in the direction of \(\overrightarrow{n_k}\) (normal vector).


\section{Computations}
\label{\detokenize{chapters/description/theory:computations}}
\sphinxAtStartPar
To converse  which claims that there should be  of the k\sphinxhyphen{}th panel, the equation below is set:

\sphinxAtStartPar
If the velocity induced at k\sphinxhyphen{}th panel is \(\overrightarrow{q_{ind}}\), no flow through the surface (boundary condition) is fulfield when:
\begin{equation}\label{equation:chapters/description/theory:bc}
\begin{split}
(\overrightarrow{u_{\infty}} + \overrightarrow{w_k})\cdot\,\overrightarrow{n_k}=0
\end{split}
\end{equation}
\sphinxAtStartPar
After equation transformation:
\begin{equation}\label{equation:chapters/description/theory:bc2}
\begin{split}
\sum_{j} \overrightarrow{\nu_{kj}}  \, \Gamma_{j} \cdot \overrightarrow{n_k}=-\overrightarrow{u_{\infty}}\,\cdot\,\overrightarrow{n_k}
\end{split}
\end{equation}
\sphinxAtStartPar
where \(\overrightarrow{\nu_{kj}}\) is defined as coefficient of proportionality of induced velocity \(\overrightarrow{q_{ind}}\) at k\sphinxhyphen{}th control point by j\sphinxhyphen{}th vortex.

\sphinxAtStartPar
By expanding equation \eqref{equation:chapters/description/theory:bc} and \eqref{equation:chapters/description/theory:bc2} the RHS coefficient vector can be computed:
\begin{equation*}
\begin{split}
RHS_k=-\overrightarrow{u_{\infty}}\,\cdot\overrightarrow{n_k}
\end{split}
\end{equation*}
\sphinxAtStartPar
In order to obtain gamma magnitude at k\sphinxhyphen{}th panel, the following set of algebraic equations must be solved:
\label{equation:chapters/description/theory:47ff99fa-4b2c-48b1-be9c-762f880dc5d4}\begin{gather}
    \begin{bmatrix} 
    a_{11} & \dots  & a_{1m}\\
    \vdots & \ddots & \vdots\\
    a_{m1} & \dots  & a_{mm}
    \end{bmatrix}
\begin{bmatrix} 
    \Gamma_{1} \\
    \vdots \\
    \Gamma_{m}
    \end{bmatrix}
 =
  \begin{bmatrix} 
    RHS_{1} \\
    \vdots \\
    RHS_{m}
    \end{bmatrix}
\end{gather}
\sphinxAtStartPar
where \(m=n_{spanwise}\cdot\,n_{chordwise}\) and \(a_{kj} = \overrightarrow{\nu_{kj}} n_k\)

\sphinxAtStartPar
The aerodynamics force can be expressed according to the Kutta\sphinxhyphen{}Joukowski theorem as:
\begin{equation*}
\begin{split}
\overrightarrow{F_{areo}} = \rho \overrightarrow{u_{\infty}} \times \overrightarrow{b} \Gamma
\end{split}
\end{equation*}
\sphinxAtStartPar
Where \(\overrightarrow{b}\) is span vector.

\sphinxAtStartPar
Discretizing, the aerodunamic force corresponding to the j\sphinxhyphen{}th section is:
\begin{equation*}
\begin{split}
\overrightarrow{F_{j}}=\rho\,\overrightarrow{V}_{app\_wind\_fs\_j} \times \overrightarrow{b_j}\Gamma_{j} = \\
\rho(\overrightarrow{V}_{app\_wind\_infs\_j} + \overrightarrow{q}_{ind\_j}) \times \overrightarrow{b_j}\Gamma_{j} = \\
\rho(\overrightarrow{V}_{app\_wind\_infs\_j} +
\sum{\overrightarrow{\nu_{jk}}\Gamma_{k}}) \times \overrightarrow{b_j}\Gamma_{j}
\end{split}
\end{equation*}
\sphinxAtStartPar
where \(b_j\) is a vector representing a finite vortex filement going through the center of pressure of the j\sphinxhyphen{}th section, \(\overrightarrow{V}_{app\_wind\_infs\_j}\) is apparent wind velocity for an ‘infinite sail’ (without induced wind velocity) and \(\overrightarrow{V}_{app\_wind\_fs\_j}\) is apparent wind velocity for a finite sail’ (with induced wind velocity).

\sphinxAtStartPar
Lift is defined as the component of the aerodynamic force that is perpendicular to the flow direction (\(\overrightarrow{u_{\infty}}\)) and can is defined as a dot product of force and normal vector:
\begin{equation*}
\begin{split}
L_k=\overrightarrow{F_k}\,\cdot \overrightarrow{n_k}
\end{split}
\end{equation*}
\sphinxAtStartPar
Pressure at k\sphinxhyphen{}th panel:
\begin{equation*}
\begin{split}
p_k=\frac{L_k}{S_k}
\end{split}
\end{equation*}
\sphinxAtStartPar
Pressure coefficient at k\sphinxhyphen{}th panel can be defined as:
\begin{equation*}
\begin{split}
c_p=\frac{p_k}{\frac{1}{2}\, \rho {\left|\left|\overrightarrow{u_{\infty_k}}\right|\right|}^2}
\end{split}
\end{equation*}








\renewcommand{\indexname}{Index}
\printindex
\end{document}