%% Generated by Sphinx.
\def\sphinxdocclass{jupyterBook}
\documentclass[a4paper,12pt,english]{jupyterBook}
\ifdefined\pdfpxdimen
   \let\sphinxpxdimen\pdfpxdimen\else\newdimen\sphinxpxdimen
\fi \sphinxpxdimen=.75bp\relax
\ifdefined\pdfimageresolution
    \pdfimageresolution= \numexpr \dimexpr1in\relax/\sphinxpxdimen\relax
\fi
%% let collapsible pdf bookmarks panel have high depth per default
\PassOptionsToPackage{bookmarksdepth=5}{hyperref}
%% turn off hyperref patch of \index as sphinx.xdy xindy module takes care of
%% suitable \hyperpage mark-up, working around hyperref-xindy incompatibility
\PassOptionsToPackage{hyperindex=false}{hyperref}
%% memoir class requires extra handling
\makeatletter\@ifclassloaded{memoir}
{\ifdefined\memhyperindexfalse\memhyperindexfalse\fi}{}\makeatother

\PassOptionsToPackage{warn}{textcomp}

\catcode`^^^^00a0\active\protected\def^^^^00a0{\leavevmode\nobreak\ }
\usepackage{cmap}
\usepackage{fontspec}
\defaultfontfeatures[\rmfamily,\sffamily,\ttfamily]{}
\usepackage{amsmath,amssymb,amstext}
\usepackage{polyglossia}
\setmainlanguage{english}



\setmainfont{FreeSerif}[
  Extension      = .otf,
  UprightFont    = *,
  ItalicFont     = *Italic,
  BoldFont       = *Bold,
  BoldItalicFont = *BoldItalic
]
\setsansfont{FreeSans}[
  Extension      = .otf,
  UprightFont    = *,
  ItalicFont     = *Oblique,
  BoldFont       = *Bold,
  BoldItalicFont = *BoldOblique,
]
\setmonofont{FreeMono}[
  Extension      = .otf,
  UprightFont    = *,
  ItalicFont     = *Oblique,
  BoldFont       = *Bold,
  BoldItalicFont = *BoldOblique,
]



\usepackage[Bjarne]{fncychap}
\usepackage[,numfigreset=1,mathnumfig]{sphinx}

\fvset{fontsize=\small}
\usepackage{geometry}
\usepackage{nicefrac, xfrac}


% Include hyperref last.
\usepackage{hyperref}
% Fix anchor placement for figures with captions.
\usepackage{hypcap}% it must be loaded after hyperref.
% Set up styles of URL: it should be placed after hyperref.
\urlstyle{same}


\usepackage{sphinxmessages}



        % Start of preamble defined in sphinx-jupyterbook-latex %
         \usepackage[Latin,Greek]{ucharclasses}
        \usepackage{unicode-math}
        % fixing title of the toc
        \addto\captionsenglish{\renewcommand{\contentsname}{Contents}}
        \hypersetup{
            pdfencoding=auto,
            psdextra
        }
        % End of preamble defined in sphinx-jupyterbook-latex %
        

\title{Future outlooks}
\date{Jun 09, 2023}
\release{}
\author{Zuzanna Wieczorek, Grzegorz Gruszczyński}
\newcommand{\sphinxlogo}{\vbox{}}
\renewcommand{\releasename}{}
\makeindex
\begin{document}

\pagestyle{empty}
\sphinxmaketitle
\clearpage

\pagestyle{plain}
\sphinxtableofcontents
\pagestyle{normal}
\phantomsection\label{\detokenize{chapters/future::doc}}


\sphinxAtStartPar
Some improvements to pySailingVLM code can be conducted in order to decrease program time execution.


\part{GPU}
\label{\detokenize{chapters/future:gpu}}
\sphinxAtStartPar
Apart form CPU programing, Numba supports CUDA GPU programming by directly compiling a restricted subset of Python code into CUDA kernels and device functions using the CUDA execution model. Kernels which are written in Numba appear to have direct access to NumPy arrays according to documantation {[}\hyperlink{cite.chapters/bibliography:id14}{num}{]}. The NumPy arrays are automatically shifted between the CPU and the GPU. To increase code performance, calcuations can be moved to GPU.


\part{All code as SoA}
\label{\detokenize{chapters/future:all-code-as-soa}}
\sphinxAtStartPar
Up to now, most of Python objects are represented as Structure of Arrays, except panels. To take full adventage of GPU speed up, panels can be rewriten into SoA. This memory layout is recommended for CUDA acceleration {[}\hyperlink{cite.chapters/bibliography:id13}{WP16}{]} due to parallel acces to data units.


\part{Points dupllicate}
\label{\detokenize{chapters/future:points-dupllicate}}
\sphinxAtStartPar
Panels which are close to each other share same points (see figure \hyperref[\detokenize{chapters/future:future-panels}]{\ref{\detokenize{chapters/future:future-panels}}}). Current pySailingVLM implementation do not take into account this fact and there are duplicates of points coordinates in panel array. Reducing redundant points can improve the code efficiency.

\begin{figure}[htbp]
\centering
\capstart

\noindent\sphinxincludegraphics[height=500\sphinxpxdimen]{{future_drawio}.png}
\caption{Points on panels arragement. Figure created by author.}\label{\detokenize{chapters/future:future-panels}}\end{figure}







\renewcommand{\indexname}{Index}
\printindex
\end{document}