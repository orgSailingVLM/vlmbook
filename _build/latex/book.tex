%% Generated by Sphinx.
\def\sphinxdocclass{jupyterBook}
\documentclass[letterpaper,10pt,english]{jupyterBook}
\ifdefined\pdfpxdimen
   \let\sphinxpxdimen\pdfpxdimen\else\newdimen\sphinxpxdimen
\fi \sphinxpxdimen=.75bp\relax
\ifdefined\pdfimageresolution
    \pdfimageresolution= \numexpr \dimexpr1in\relax/\sphinxpxdimen\relax
\fi
%% let collapsible pdf bookmarks panel have high depth per default
\PassOptionsToPackage{bookmarksdepth=5}{hyperref}
%% turn off hyperref patch of \index as sphinx.xdy xindy module takes care of
%% suitable \hyperpage mark-up, working around hyperref-xindy incompatibility
\PassOptionsToPackage{hyperindex=false}{hyperref}
%% memoir class requires extra handling
\makeatletter\@ifclassloaded{memoir}
{\ifdefined\memhyperindexfalse\memhyperindexfalse\fi}{}\makeatother

\PassOptionsToPackage{warn}{textcomp}

\catcode`^^^^00a0\active\protected\def^^^^00a0{\leavevmode\nobreak\ }
\usepackage{cmap}
\usepackage{fontspec}
\defaultfontfeatures[\rmfamily,\sffamily,\ttfamily]{}
\usepackage{amsmath,amssymb,amstext}
\usepackage{polyglossia}
\setmainlanguage{english}



\setmainfont{FreeSerif}[
  Extension      = .otf,
  UprightFont    = *,
  ItalicFont     = *Italic,
  BoldFont       = *Bold,
  BoldItalicFont = *BoldItalic
]
\setsansfont{FreeSans}[
  Extension      = .otf,
  UprightFont    = *,
  ItalicFont     = *Oblique,
  BoldFont       = *Bold,
  BoldItalicFont = *BoldOblique,
]
\setmonofont{FreeMono}[
  Extension      = .otf,
  UprightFont    = *,
  ItalicFont     = *Oblique,
  BoldFont       = *Bold,
  BoldItalicFont = *BoldOblique,
]



\usepackage[Bjarne]{fncychap}
\usepackage[,numfigreset=1,mathnumfig]{sphinx}

\fvset{fontsize=\small}
\usepackage{geometry}


% Include hyperref last.
\usepackage{hyperref}
% Fix anchor placement for figures with captions.
\usepackage{hypcap}% it must be loaded after hyperref.
% Set up styles of URL: it should be placed after hyperref.
\urlstyle{same}

\addto\captionsenglish{\renewcommand{\contentsname}{Description}}

\usepackage{sphinxmessages}



        % Start of preamble defined in sphinx-jupyterbook-latex %
         \usepackage[Latin,Greek]{ucharclasses}
        \usepackage{unicode-math}
        % fixing title of the toc
        \addto\captionsenglish{\renewcommand{\contentsname}{Contents}}
        \hypersetup{
            pdfencoding=auto,
            psdextra
        }
        % End of preamble defined in sphinx-jupyterbook-latex %
        

\title{pySailingVLm Documentation}
\date{Apr 29, 2023}
\release{}
\author{Grzegorz Gruszczyński, Zuzanna Wieczorek}
\newcommand{\sphinxlogo}{\vbox{}}
\renewcommand{\releasename}{}
\makeindex
\begin{document}

\pagestyle{empty}
\sphinxmaketitle
\pagestyle{plain}
\sphinxtableofcontents
\pagestyle{normal}
\phantomsection\label{\detokenize{chapters/intro::doc}}


\sphinxAtStartPar
The Vortex lattice method, also called VLM, is a numerical method used in computational fluid dynamics. VLM models a surface on aircraft as infinite vortices to estimate the lift curve slope, induced drag, and force distribution. The VLM is the extension of Prandtl’s lifting\sphinxhyphen{}line theory that is capable of computing swept and low aspect ratio wings. {[}\hyperlink{cite.chapters/bibliography:id7}{SHR+19}{]}.

\sphinxAtStartPar
In this thesis, the method was used to determinate aerodynamic parameters such as CL, CD  and lift\sphinxhyphen{}to\sphinxhyphen{}drag ratio for modeling behaviour of sailing yacht with different initial wind conditions.
The main purpose of this work was to update existing VLM code to be faster and more efficient. To achieve this goal authors have changed the structure of data and applied some compilations for specific functions in Python code using Numba which translates a subset of Python and NumPy code into fast machine code. Apart from this, the code has gained some bug fixes and new features such as cambered sail, pressure coefficient colormap, validation tests, command line script and jupyter notebook examples. Moreover, code has been packaged and now is available at The Python Package Index (PyPI) \sphinxhyphen{} a repository of software for the Python programming language. \sphinxhref{https://jupyterbook.org}{\#TODO pySailingVLM at PyPI}

\sphinxAtStartPar
Keywords: Vortex Lattice Method, VLM, sailing, yacht, Python, code, visualisation, yacht engineering

\sphinxAtStartPar
Inside documentation:
\begin{itemize}
\item {} 
\sphinxAtStartPar
Description

\begin{itemize}
\item {} 
\sphinxAtStartPar
{\hyperref[\detokenize{chapters/theory::doc}]{\sphinxcrossref{Theory}}}

\item {} 
\sphinxAtStartPar
{\hyperref[\detokenize{chapters/technology::doc}]{\sphinxcrossref{Technology}}}

\end{itemize}
\end{itemize}
\begin{itemize}
\item {} 
\sphinxAtStartPar
Getting Started

\begin{itemize}
\item {} 
\sphinxAtStartPar
{\hyperref[\detokenize{chapters/installation::doc}]{\sphinxcrossref{Installation}}}

\item {} 
\sphinxAtStartPar
{\hyperref[\detokenize{chapters/usage::doc}]{\sphinxcrossref{Usage}}}

\item {} 
\sphinxAtStartPar
{\hyperref[\detokenize{chapters/examples::doc}]{\sphinxcrossref{Examples}}}

\item {} 
\sphinxAtStartPar
{\hyperref[\detokenize{chapters/validation::doc}]{\sphinxcrossref{Validation}}}

\item {} 
\sphinxAtStartPar
{\hyperref[\detokenize{chapters/notebooks::doc}]{\sphinxcrossref{Content with notebooks}}}

\item {} 
\sphinxAtStartPar
{\hyperref[\detokenize{chapters/bibliography::doc}]{\sphinxcrossref{bibliography}}}

\end{itemize}
\end{itemize}

\sphinxstepscope


\part{Description}

\sphinxstepscope


\chapter{Theory}
\label{\detokenize{chapters/theory:theory}}\label{\detokenize{chapters/theory::doc}}

\section{Introduction}
\label{\detokenize{chapters/theory:introduction}}
\sphinxAtStartPar
Vortex Lattice Method is build on the theory of ideal flow, known as Potential flow. Ideal flow is a simplication of the real flow which can be seen in nature. This  approximation is suffiecient from the engineering point of view. VLM do not take into account any turbulence, dissipation and viscous effects.


\section{Potential flow theory}
\label{\detokenize{chapters/theory:potential-flow-theory}}
\sphinxAtStartPar
Potenttial Flow Theory treats external flows around bodies as invicid and irrational {[}\hyperlink{cite.chapters/bibliography:id5}{Tec05}{]} .The viscous effects are limitted to a thin layer next to the body (boundary layer). Becouse of this and separation phenomena such fluid can be modelled only with small angle of attack. In irrational flow fluid particles are not rotating.

\sphinxAtStartPar
In Potential Flow Theory, because velocity must satisfy the conservation of mass equation, the Laplace Equation is qoverning:
\label{equation:chapters/theory:ebe07e2b-f09a-40d3-ac4f-4dba96e79d3a}\begin{equation}
\label{nabla}
\nabla^{2} \phi = 0
\end{equation}
\sphinxAtStartPar
where \$\textbackslash{}phi\$ is a potentinal function defined as continous function which satisfies the conservation of mass and momentum (incompressible, inviscid and irrational flow).


\chapter{Vortex Lattice method}
\label{\detokenize{chapters/theory:vortex-lattice-method}}
\sphinxAtStartPar
The Vortex Lattice Method is a panel method where wing or other configuration is modelled by a large number of elementary, quadrilateral panels lying either on the actual aircraft surface, or on some mean surface, or combination thereof. Sources, vortices and doublets are attached to each panel. Such singularities are defined by specifying functional variation across the panel and its value is set by determinating strength parameters. Such parameters are known after solving appropriate boundary condition equations. When singularity strengts are determinated, the pressure, velocity can be computed{[}\hyperlink{cite.chapters/bibliography:id4}{JJB09}{]}.

\begin{figure}[htbp]
\centering
\capstart

\noindent\sphinxincludegraphics[height=250\sphinxpxdimen]{{panels}.png}
\caption{Representation of an airplane flowfield by panel (or singularity) methods. Figure taken from {[}\hyperlink{cite.chapters/bibliography:id4}{JJB09}{]} (Fig. 7.22 page 350).}\label{\detokenize{chapters/theory:panel-method}}\end{figure}


\section{Vortices}
\label{\detokenize{chapters/theory:vortices}}
\sphinxAtStartPar
Element which creates circulation around its origin is called a vortex. According to German scientist Hermann von Helmholtz who developed vortex theorems for inviscid incompressible flows {[}\hyperlink{cite.chapters/bibliography:id3}{JK01}{]}:
\begin{itemize}
\item {} 
\sphinxAtStartPar
The strength of vortex filament is constant along its length

\item {} 
\sphinxAtStartPar
A vortex filament cannot start or end in a fluid (it must form or closed path or extend to infinity)

\item {} 
\sphinxAtStartPar
The fluid that forms a vortex tube continues to form a vortex tube and the strength of the vortex tube remains constant as the tube moves about.

\end{itemize}


\subsection{Vortex element}
\label{\detokenize{chapters/theory:vortex-element}}
\sphinxAtStartPar
Vortex element in 2D is presented on figure \hyperref[\detokenize{chapters/theory:vortex}]{\ref{\detokenize{chapters/theory:vortex}}}. It is created by placing a rotating cylindrical core in two dimentional flowfield. If the cylinder has a radius \$R\$ and a constant angular velocity of \$\textbackslash{}omega\_y\$, then its motion results in a flow with circular streamlines {[}\hyperlink{cite.chapters/bibliography:id3}{JK01}{]}.

\begin{figure}[htbp]
\centering
\capstart

\noindent\sphinxincludegraphics[scale=1.0]{{vort}.png}
\caption{Two dimentional flowfield around a cylindrical core rotating as a rigid body. Figure taken from {[}\hyperlink{cite.chapters/bibliography:id3}{JK01}{]} (Fig. 2.11 page 34).}\label{\detokenize{chapters/theory:vortex}}\end{figure}

\sphinxAtStartPar
The tangential velocity induced is describe by equation:
\label{equation:chapters/theory:819b5d19-2c73-42a3-9cfd-33e222846ab5}\begin{equation}
\label{q_theta}
q_\theta(r) = \frac{\Gamma}{2\pi \cdot r}
\end{equation}
\sphinxAtStartPar
where \$r\$ is described as distance from the vortex.

\sphinxAtStartPar
If a vortex is located in free\sphinxhyphen{}strem with uniform velocity \$u\_\textbackslash{}infty\$ then the total velocity at the distance \$r\$ can be writtes as \$u\_\textbackslash{}infty + q\_\textbackslash{}theta(r)\$ {[}\hyperlink{cite.chapters/bibliography:id2}{AH13}{]}.


\subsection{Vortex segment}
\label{\detokenize{chapters/theory:vortex-segment}}
\begin{figure}[htbp]
\centering
\capstart

\noindent\sphinxincludegraphics[scale=1.0]{{vor_seg}.png}
\caption{Vortex segment in three dimentions. Figure taken from {[}\hyperlink{cite.chapters/bibliography:id3}{JK01}{]} (Fig. 2.16 page 40).}\label{\detokenize{chapters/theory:id9}}\end{figure}

\sphinxAtStartPar
The velocity induced by a straight vortex segement (figure \hyperref[\detokenize{chapters/theory:id9}]{\ref{\detokenize{chapters/theory:id9}}}) in three dimentions with given vortex strength \$\textbackslash{}Gamma\$ at point \$P(x,y,z)\$ is qiven by equation:
\label{equation:chapters/theory:58a44f7b-7737-4cce-ab01-fc9054bfd312}\begin{equation}
\label{eq:q12}
q_{1,2} = \frac{\Gamma}{4\pi}\frac{\overrightarrow{r_1} \times \overrightarrow{r_2}}{|\overrightarrow{r_1} \times \overrightarrow{r_2}|^2} \left(\frac{\overrightarrow{r_1}}{||\overrightarrow{r_1}||} - \frac{\overrightarrow{r_2}}{||\overrightarrow{r_2}||}\right) \cdot\overrightarrow{r_0}
\end{equation}
\sphinxAtStartPar
where
\label{equation:chapters/theory:9d118334-5810-40d3-8fc8-1b855eae0d07}\begin{equation}
\overrightarrow{r_{0}} = \overrightarrow{r_1} - \overrightarrow{r_2}
\end{equation}
\sphinxAtStartPar
and
\label{equation:chapters/theory:cda939b1-2a1b-4b3c-9087-d88575861473}\begin{gather}
\overrightarrow{r_{0}} = (x_2, y_2, z_2) - (x_1, y_1, z_1) \\
\overrightarrow{r_{1}} = P - (x_1, y_1, z_1)\\
\overrightarrow{r_{2}} = P - (x_2, y_2, z_2)
\end{gather}
\sphinxAtStartPar
In case of a semi\sphinxhyphen{}infinite vortex line, the point 2 is infinitely far away thus formula reads {[}\hyperlink{cite.chapters/bibliography:id6}{PS00}{]}:
\label{equation:chapters/theory:59c102ad-605b-4455-8c38-ed1aed1d0a90}\begin{equation}
\label{q12_inf}
q_{1,2} = \frac{\Gamma}{4\pi}\frac{\overrightarrow{u_\infty} \times \overrightarrow{r_1}}{||\overrightarrow{r_1}||(||\overrightarrow{r_1}|| - \overrightarrow{u_\infty} \cdot \overrightarrow{r_1})} 
\end{equation}
\sphinxAtStartPar
Referencing do wyrazen matematycznych dziala w latexie, w htmlu nie, wiec rzeba tak pisac zdania zeby w latxie byly odnosniki a w html nie bylo to widoczne np tak jak w ostatnim wyrazeniu matematycznym \ref{eq:q12}.

\sphinxstepscope


\chapter{Technology}
\label{\detokenize{chapters/technology:technology}}\label{\detokenize{chapters/technology::doc}}
\sphinxstepscope


\part{Getting Started}

\sphinxstepscope


\chapter{Installation}
\label{\detokenize{chapters/installation:installation}}\label{\detokenize{chapters/installation::doc}}

\section{For Users}
\label{\detokenize{chapters/installation:for-users}}
\sphinxAtStartPar
pySailingVLM package is available at PyPI, to install it do:

\begin{sphinxVerbatim}[commandchars=\\\{\}]
\PYG{n}{pip} \PYG{n}{install} \PYG{n}{pySailingVLM}
\end{sphinxVerbatim}


\section{For Developers}
\label{\detokenize{chapters/installation:for-developers}}
\sphinxAtStartPar
If you would like to dive into pySailingVLM code and test it on your own ,please do the folowing steps:
\begin{itemize}
\item {} 
\sphinxAtStartPar
Clone git project

\end{itemize}

\begin{sphinxVerbatim}[commandchars=\\\{\}]
\PYG{n}{git} \PYG{n}{clone} \PYG{c+c1}{\PYGZsh{}TODO wsadzic tutaj link}
\end{sphinxVerbatim}
\begin{itemize}
\item {} 
\sphinxAtStartPar
Go into project

\end{itemize}

\begin{sphinxVerbatim}[commandchars=\\\{\}]
\PYG{n}{cd} \PYG{c+c1}{\PYGZsh{}TODO project}
\end{sphinxVerbatim}
\begin{itemize}
\item {} 
\sphinxAtStartPar
There is no requirements.txt file, for installing dependencies install a project in editable mode (i.e. setuptools “develop mode”) from a local project path:

\end{itemize}

\begin{sphinxVerbatim}[commandchars=\\\{\}]
\PYG{n}{pip} \PYG{n}{install} \PYG{o}{\PYGZhy{}}\PYG{n}{q} \PYG{o}{\PYGZhy{}}\PYG{n}{e} \PYG{o}{.}
\end{sphinxVerbatim}


\subsection{Create package}
\label{\detokenize{chapters/installation:create-package}}
\sphinxAtStartPar
Building package requires setup.cfg and pyproject.toml files located in main tree of pySailingVLM project. If you do not have the latest pip build package do:

\begin{sphinxVerbatim}[commandchars=\\\{\}]
\PYG{n}{pip} \PYG{n}{install} \PYG{o}{\PYGZhy{}}\PYG{o}{\PYGZhy{}}\PYG{n}{upgrade} \PYG{n}{build}
\end{sphinxVerbatim}

\sphinxAtStartPar
Then:

\begin{sphinxVerbatim}[commandchars=\\\{\}]
\PYG{n}{pip} \PYG{n}{install} \PYG{n}{build}
\end{sphinxVerbatim}


\subsubsection{Build and install}
\label{\detokenize{chapters/installation:build-and-install}}
\begin{sphinxVerbatim}[commandchars=\\\{\}]
\PYG{n}{python3} \PYG{o}{\PYGZhy{}}\PYG{n}{m} \PYG{n}{build}
\end{sphinxVerbatim}

\begin{sphinxVerbatim}[commandchars=\\\{\}]
\PYG{n}{pip} \PYG{n}{install} \PYG{n}{dist}\PYG{o}{/}\PYG{n}{pySailingVLM}\PYG{o}{\PYGZhy{}}\PYG{n}{VERSION}\PYG{o}{.}\PYG{n}{tar}\PYG{o}{.}\PYG{n}{gz}
\end{sphinxVerbatim}

\sphinxstepscope


\chapter{Usage}
\label{\detokenize{chapters/usage:usage}}\label{\detokenize{chapters/usage::doc}}
\begin{sphinxadmonition}{note}{Note:}
\sphinxAtStartPar
First usage of pySailingVLm will produce numba warining. This behaviour is correct. Warining messages will disappear during second run of program.

\sphinxAtStartPar
Example warning:

\sphinxAtStartPar
/home/zuzanna/Documents/miniconda3/envs/sv\_build\_test\_2/lib/python3.10/site\sphinxhyphen{}packages/numba/core/lowering.py:107: NumbaDebugInfoWarning: Could not find source for function: <function \_\_numba\_array\_expr\_0x7fbf333f1780 at 0x7fbf33361b40>. Debug line information may be inaccurate.
\end{sphinxadmonition}


\section{Run from command line}
\label{\detokenize{chapters/usage:run-from-command-line}}

\subsection{input file}
\label{\detokenize{chapters/usage:input-file}}
\sphinxAtStartPar
In order to run pySailingVLM from command line you must provide a variable.py file. Example file is shown below. Modify it for your needs.

\begin{sphinxVerbatim}[commandchars=\\\{\}]
\PYG{k+kn}{import} \PYG{n+nn}{os}
\PYG{k+kn}{import} \PYG{n+nn}{numpy} \PYG{k}{as} \PYG{n+nn}{np}
\PYG{k+kn}{import} \PYG{n+nn}{time}

\PYG{c+c1}{\PYGZsh{} OUTPUT DIR}
\PYG{n}{case\PYGZus{}name} \PYG{o}{=} \PYG{n}{os}\PYG{o}{.}\PYG{n}{path}\PYG{o}{.}\PYG{n}{basename}\PYG{p}{(}\PYG{n+nv+vm}{\PYGZus{}\PYGZus{}file\PYGZus{}\PYGZus{}}\PYG{p}{)}  \PYG{c+c1}{\PYGZsh{} get name of the current file}
\PYG{n}{case\PYGZus{}dir} \PYG{o}{=} \PYG{n}{os}\PYG{o}{.}\PYG{n}{path}\PYG{o}{.}\PYG{n}{dirname}\PYG{p}{(}\PYG{n}{os}\PYG{o}{.}\PYG{n}{path}\PYG{o}{.}\PYG{n}{realpath}\PYG{p}{(}\PYG{n+nv+vm}{\PYGZus{}\PYGZus{}file\PYGZus{}\PYGZus{}}\PYG{p}{)}\PYG{p}{)}  \PYG{c+c1}{\PYGZsh{} get dir of the current file}
\PYG{n}{time\PYGZus{}stamp} \PYG{o}{=} \PYG{n}{time}\PYG{o}{.}\PYG{n}{strftime}\PYG{p}{(}\PYG{l+s+s2}{\PYGZdq{}}\PYG{l+s+s2}{\PYGZpc{}}\PYG{l+s+s2}{Y\PYGZhy{}}\PYG{l+s+s2}{\PYGZpc{}}\PYG{l+s+s2}{m\PYGZhy{}}\PYG{l+s+si}{\PYGZpc{}d}\PYG{l+s+s2}{\PYGZus{}}\PYG{l+s+s2}{\PYGZpc{}}\PYG{l+s+s2}{Hh}\PYG{l+s+s2}{\PYGZpc{}}\PYG{l+s+s2}{Mm}\PYG{l+s+s2}{\PYGZpc{}}\PYG{l+s+s2}{Ss}\PYG{l+s+s2}{\PYGZdq{}}\PYG{p}{)}
\PYG{n}{output\PYGZus{}dir\PYGZus{}name} \PYG{o}{=} \PYG{n}{os}\PYG{o}{.}\PYG{n}{path}\PYG{o}{.}\PYG{n}{join}\PYG{p}{(}\PYG{l+s+s2}{\PYGZdq{}}\PYG{l+s+s2}{results\PYGZus{}example\PYGZus{}jib\PYGZus{}and\PYGZus{}mainsail\PYGZus{}vlm}\PYG{l+s+s2}{\PYGZdq{}}\PYG{p}{,} \PYG{n}{time\PYGZus{}stamp}\PYG{p}{)}
\PYG{n}{file\PYGZus{}name} \PYG{o}{=} \PYG{l+s+s1}{\PYGZsq{}}\PYG{l+s+s1}{my\PYGZus{}fancy\PYGZus{}results}\PYG{l+s+s1}{\PYGZsq{}} \PYG{c+c1}{\PYGZsh{} name of xlsx excel file}

\PYG{c+c1}{\PYGZsh{} SOLVER SETTINGS}
\PYG{n}{n\PYGZus{}spanwise} \PYG{o}{=} \PYG{l+m+mi}{4}  \PYG{c+c1}{\PYGZsh{} No of control points (above the water) per sail, recommended: 50}
\PYG{n}{n\PYGZus{}chordwise} \PYG{o}{=} \PYG{l+m+mi}{1} \PYG{c+c1}{\PYGZsh{} No of control points (above the water) per sail, recommended: 50}
\PYG{n}{interpolation\PYGZus{}type} \PYG{o}{=} \PYG{l+s+s2}{\PYGZdq{}}\PYG{l+s+s2}{linear}\PYG{l+s+s2}{\PYGZdq{}}  \PYG{c+c1}{\PYGZsh{} either \PYGZdq{}spline\PYGZdq{} or \PYGZdq{}linear\PYGZdq{}}
\PYG{n}{LLT\PYGZus{}twist} \PYG{o}{=} \PYG{l+s+s2}{\PYGZdq{}}\PYG{l+s+s2}{real\PYGZus{}twist}\PYG{l+s+s2}{\PYGZdq{}}  \PYG{c+c1}{\PYGZsh{} defines how the Lifting Line discretize the sail twist.}
\PYG{c+c1}{\PYGZsh{} It can be \PYGZdq{}sheeting\PYGZus{}angle\PYGZus{}const\PYGZdq{} or \PYGZdq{}average\PYGZus{}const\PYGZdq{} or \PYGZdq{}real\PYGZus{}twist\PYGZdq{}}

\PYG{c+c1}{\PYGZsh{} SAILING CONDITIONS}
\PYG{n}{leeway\PYGZus{}deg} \PYG{o}{=} \PYG{l+m+mf}{0.}    \PYG{c+c1}{\PYGZsh{} [deg]}
\PYG{n}{heel\PYGZus{}deg} \PYG{o}{=} \PYG{l+m+mf}{0.}     \PYG{c+c1}{\PYGZsh{} [deg]}
\PYG{n}{SOG\PYGZus{}yacht} \PYG{o}{=} \PYG{l+m+mf}{0.}   \PYG{c+c1}{\PYGZsh{} [m/s] yacht speed \PYGZhy{} speed over ground (leeway is a separate variable)}
\PYG{n}{tws\PYGZus{}ref} \PYG{o}{=} \PYG{l+m+mf}{1.}     \PYG{c+c1}{\PYGZsh{} [m/s] true wind speed}
\PYG{n}{alpha\PYGZus{}true\PYGZus{}wind\PYGZus{}deg} \PYG{o}{=} \PYG{l+m+mi}{0}\PYG{c+c1}{\PYGZsh{}10.   \PYGZsh{} [deg] true wind angle (with reference to course over ground) =\PYGZgt{} Course Wind Angle to the boat track = true wind angle to centerline + Leeway}
\PYG{n}{reference\PYGZus{}water\PYGZus{}level\PYGZus{}for\PYGZus{}wind\PYGZus{}profile} \PYG{o}{=} \PYG{l+m+mf}{0.}  \PYG{c+c1}{\PYGZsh{} [m] this is an attempt to mimick the deck effect}
\PYG{c+c1}{\PYGZsh{} by lowering the sheer\PYGZus{}above\PYGZus{}waterline}
\PYG{c+c1}{\PYGZsh{} while keeping the wind profile as in original geometry}
\PYG{c+c1}{\PYGZsh{} this shall be negative (H = sail\PYGZus{}ctrl\PYGZus{}point \PYGZhy{} water\PYGZus{}level)}
\PYG{n}{wind\PYGZus{}exp\PYGZus{}coeff} \PYG{o}{=} \PYG{l+m+mf}{0.}  \PYG{c+c1}{\PYGZsh{} [\PYGZhy{}] coefficient to determine the exponential wind profile}
\PYG{n}{wind\PYGZus{}reference\PYGZus{}measurment\PYGZus{}height} \PYG{o}{=} \PYG{l+m+mf}{10.}  \PYG{c+c1}{\PYGZsh{} [m] reference height for exponential wind profile}
\PYG{n}{rho} \PYG{o}{=} \PYG{l+m+mf}{1.225}  \PYG{c+c1}{\PYGZsh{} air density [kg/m3]}
\PYG{n}{wind\PYGZus{}profile} \PYG{o}{=} \PYG{l+s+s1}{\PYGZsq{}}\PYG{l+s+s1}{flat}\PYG{l+s+s1}{\PYGZsq{}} \PYG{c+c1}{\PYGZsh{} allowed: \PYGZsq{}exponential\PYGZsq{} or \PYGZsq{}flat\PYGZsq{} or \PYGZsq{}logarithmic\PYGZsq{}}
\PYG{n}{roughness} \PYG{o}{=} \PYG{l+m+mf}{0.05} \PYG{c+c1}{\PYGZsh{} for logarithmic profile only}

\PYG{c+c1}{\PYGZsh{} GEOMETRY OF THE RIG}
\PYG{n}{main\PYGZus{}sail\PYGZus{}luff} \PYG{o}{=} \PYG{l+m+mf}{0.5}\PYG{c+c1}{\PYGZsh{}10. \PYGZsh{} 12.4  \PYGZsh{} [m]}
\PYG{n}{jib\PYGZus{}luff} \PYG{o}{=} \PYG{l+m+mf}{10.0}  \PYG{c+c1}{\PYGZsh{} [m]}
\PYG{n}{foretriangle\PYGZus{}height} \PYG{o}{=} \PYG{l+m+mf}{11.50}  \PYG{c+c1}{\PYGZsh{} [m]}
\PYG{n}{foretriangle\PYGZus{}base} \PYG{o}{=} \PYG{l+m+mf}{3.90}  \PYG{c+c1}{\PYGZsh{} [m]}
\PYG{n}{sheer\PYGZus{}above\PYGZus{}waterline} \PYG{o}{=} \PYG{l+m+mf}{0.}  \PYG{c+c1}{\PYGZsh{} [m]}
\PYG{n}{boom\PYGZus{}above\PYGZus{}sheer} \PYG{o}{=} \PYG{l+m+mf}{0.} \PYG{c+c1}{\PYGZsh{}1.3  \PYGZsh{} [m]}
\PYG{n}{rake\PYGZus{}deg} \PYG{o}{=} \PYG{l+m+mi}{45}\PYG{o}{+}\PYG{l+m+mf}{90.}  \PYG{c+c1}{\PYGZsh{} rake angle [deg]}
\PYG{n}{mast\PYGZus{}LOA} \PYG{o}{=} \PYG{l+m+mf}{0.0}  \PYG{c+c1}{\PYGZsh{} [m]}

\PYG{c+c1}{\PYGZsh{} INPUT \PYGZhy{} GEOMETRY OF THE SAIL}
\PYG{n}{sails\PYGZus{}def} \PYG{o}{=} \PYG{l+s+s1}{\PYGZsq{}}\PYG{l+s+s1}{main}\PYG{l+s+s1}{\PYGZsq{}} \PYG{c+c1}{\PYGZsh{} definition of sail set, possible: \PYGZsq{}jib\PYGZsq{} or \PYGZsq{}main\PYGZsq{} or \PYGZsq{}jib\PYGZus{}and\PYGZus{}main\PYGZsq{}}
\PYG{c+c1}{\PYGZsh{} sails\PYGZus{}def = \PYGZsq{}main\PYGZsq{}}
\PYG{n}{main\PYGZus{}sail\PYGZus{}girths} \PYG{o}{=} \PYG{n}{np}\PYG{o}{.}\PYG{n}{array}\PYG{p}{(}\PYG{p}{[}\PYG{l+m+mf}{0.00}\PYG{p}{,} \PYG{l+m+mf}{1.}\PYG{o}{/}\PYG{l+m+mi}{8}\PYG{p}{,} \PYG{l+m+mf}{1.}\PYG{o}{/}\PYG{l+m+mi}{4}\PYG{p}{,} \PYG{l+m+mf}{1.}\PYG{o}{/}\PYG{l+m+mi}{2}\PYG{p}{,} \PYG{l+m+mf}{3.}\PYG{o}{/}\PYG{l+m+mi}{4}\PYG{p}{,} \PYG{l+m+mf}{7.}\PYG{o}{/}\PYG{l+m+mi}{8}\PYG{p}{,} \PYG{l+m+mf}{1.00}\PYG{p}{]}\PYG{p}{)}
\PYG{n}{main\PYGZus{}sail\PYGZus{}chords} \PYG{o}{=} \PYG{n}{np}\PYG{o}{.}\PYG{n}{array}\PYG{p}{(}\PYG{p}{[}\PYG{l+m+mf}{0.2}\PYG{p}{]}\PYG{o}{*} \PYG{n+nb}{len}\PYG{p}{(}\PYG{n}{main\PYGZus{}sail\PYGZus{}girths}\PYG{p}{)}\PYG{p}{)} \PYG{c+c1}{\PYGZsh{} np.array([4.00, 3.82, 3.64, 3.20, 2.64, 2.32, 2.00])}
\PYG{n}{main\PYGZus{}sail\PYGZus{}centerline\PYGZus{}twist\PYGZus{}deg} \PYG{o}{=} \PYG{o}{\PYGZhy{}}\PYG{l+m+mi}{2} \PYG{o}{+} \PYG{l+m+mi}{0}\PYG{o}{*} \PYG{n}{main\PYGZus{}sail\PYGZus{}girths} \PYG{c+c1}{\PYGZsh{} 10 + 12. * main\PYGZus{}sail\PYGZus{}girths  \PYGZsh{}}

\PYG{c+c1}{\PYGZsh{} First digit describing maximum camber as percentage of the chord.}
\PYG{c+c1}{\PYGZsh{} Second digit describing the distance of maximum camber from the airfoil leading edge in tenths of the chord.}
\PYG{n}{jib\PYGZus{}sail\PYGZus{}camber}\PYG{o}{=} \PYG{l+m+mi}{0}\PYG{o}{*}\PYG{n}{np}\PYG{o}{.}\PYG{n}{array}\PYG{p}{(}\PYG{p}{[}\PYG{l+m+mf}{0.01}\PYG{p}{,} \PYG{l+m+mf}{0.01}\PYG{p}{,} \PYG{l+m+mf}{0.01}\PYG{p}{,} \PYG{l+m+mf}{0.01}\PYG{p}{,} \PYG{l+m+mf}{0.01}\PYG{p}{]}\PYG{p}{)}
\PYG{n}{jib\PYGZus{}sail\PYGZus{}camber\PYGZus{}distance\PYGZus{}from\PYGZus{}luff} \PYG{o}{=} \PYG{n}{np}\PYG{o}{.}\PYG{n}{array}\PYG{p}{(}\PYG{p}{[}\PYG{l+m+mf}{0.5}\PYG{p}{,} \PYG{l+m+mf}{0.5}\PYG{p}{,} \PYG{l+m+mf}{0.5}\PYG{p}{,} \PYG{l+m+mf}{0.5}\PYG{p}{,} \PYG{l+m+mf}{0.5}\PYG{p}{]}\PYG{p}{)} \PYG{c+c1}{\PYGZsh{} starting from leading edge}
\PYG{n}{main\PYGZus{}sail\PYGZus{}camber}\PYG{o}{=} \PYG{l+m+mi}{0}\PYG{o}{*}\PYG{n}{np}\PYG{o}{.}\PYG{n}{array}\PYG{p}{(}\PYG{p}{[}\PYG{l+m+mf}{0.01}\PYG{p}{,} \PYG{l+m+mf}{0.01}\PYG{p}{,} \PYG{l+m+mf}{0.01}\PYG{p}{,} \PYG{l+m+mf}{0.01}\PYG{p}{,} \PYG{l+m+mf}{0.01}\PYG{p}{,} \PYG{l+m+mf}{0.01}\PYG{p}{,} \PYG{l+m+mf}{0.01}\PYG{p}{]}\PYG{p}{)}
\PYG{n}{main\PYGZus{}sail\PYGZus{}camber\PYGZus{}distance\PYGZus{}from\PYGZus{}luff} \PYG{o}{=} \PYG{n}{np}\PYG{o}{.}\PYG{n}{array}\PYG{p}{(}\PYG{p}{[}\PYG{l+m+mf}{0.5}\PYG{p}{,} \PYG{l+m+mf}{0.5}\PYG{p}{,} \PYG{l+m+mf}{0.5}\PYG{p}{,} \PYG{l+m+mf}{0.5}\PYG{p}{,} \PYG{l+m+mf}{0.5}\PYG{p}{,} \PYG{l+m+mf}{0.5}\PYG{p}{,} \PYG{l+m+mf}{0.5}\PYG{p}{]}\PYG{p}{)} \PYG{c+c1}{\PYGZsh{} starting from leading edge}

\PYG{n}{jib\PYGZus{}girths} \PYG{o}{=} \PYG{n}{np}\PYG{o}{.}\PYG{n}{array}\PYG{p}{(}\PYG{p}{[}\PYG{l+m+mf}{0.00}\PYG{p}{,} \PYG{l+m+mf}{1.}\PYG{o}{/}\PYG{l+m+mi}{4}\PYG{p}{,} \PYG{l+m+mf}{1.}\PYG{o}{/}\PYG{l+m+mi}{2}\PYG{p}{,} \PYG{l+m+mf}{3.}\PYG{o}{/}\PYG{l+m+mi}{4}\PYG{p}{,} \PYG{l+m+mf}{1.00}\PYG{p}{]}\PYG{p}{)}
\PYG{n}{jib\PYGZus{}chords} \PYG{o}{=} \PYG{l+m+mf}{1E\PYGZhy{}12}\PYG{o}{*} \PYG{n}{np}\PYG{o}{.}\PYG{n}{array}\PYG{p}{(}\PYG{p}{[}\PYG{l+m+mf}{3.80}\PYG{p}{,} \PYG{l+m+mf}{2.98}\PYG{p}{,} \PYG{l+m+mf}{2.15}\PYG{p}{,} \PYG{l+m+mf}{1.33}\PYG{p}{,} \PYG{l+m+mf}{0.5}\PYG{p}{]}\PYG{p}{)}
\PYG{n}{jib\PYGZus{}centerline\PYGZus{}twist\PYGZus{}deg} \PYG{o}{=}  \PYG{l+m+mi}{0}\PYG{o}{*}\PYG{p}{(}\PYG{l+m+mi}{10}\PYG{o}{+}\PYG{l+m+mi}{5}\PYG{p}{)}  \PYG{o}{+} \PYG{l+m+mi}{0}\PYG{o}{*}\PYG{l+m+mf}{15.} \PYG{o}{*} \PYG{n}{jib\PYGZus{}girths} \PYG{c+c1}{\PYGZsh{} (10+5)  + 15. * jib\PYGZus{}girths \PYGZsh{} }

\PYG{c+c1}{\PYGZsh{} REFERENCE CSYS}
\PYG{c+c1}{\PYGZsh{} The origin of the default CSYS is located @ waterline level and aft face of the mast}
\PYG{c+c1}{\PYGZsh{} The positive x\PYGZhy{}coord: towards stern}
\PYG{c+c1}{\PYGZsh{} The positive y\PYGZhy{}coord: towards leeward side}
\PYG{c+c1}{\PYGZsh{} The positive z\PYGZhy{}coord: above the water}
\PYG{c+c1}{\PYGZsh{} To shift the default CSYS, adjust the \PYGZsq{}reference\PYGZus{}level\PYGZus{}for\PYGZus{}moments\PYGZsq{} variable.}
\PYG{c+c1}{\PYGZsh{} Shifted CSYS = original + reference\PYGZus{}level\PYGZus{}for\PYGZus{}moments}
\PYG{c+c1}{\PYGZsh{} As a results the moments will be calculated around the new origin.}

\PYG{c+c1}{\PYGZsh{} yaw\PYGZus{}reference [m] \PYGZhy{} distance from the aft of the mast towards stern, at which the yawing moment is calculated.}
\PYG{c+c1}{\PYGZsh{} sway\PYGZus{}reference [m] \PYGZhy{} distance from the aft of the mast towards leeward side. 0 for symmetric yachts ;)}
\PYG{c+c1}{\PYGZsh{} heeling\PYGZus{}reference [m] \PYGZhy{} distance from the water level,  at which the heeling moment is calculated.}
\PYG{n}{reference\PYGZus{}level\PYGZus{}for\PYGZus{}moments} \PYG{o}{=} \PYG{n}{np}\PYG{o}{.}\PYG{n}{array}\PYG{p}{(}\PYG{p}{[}\PYG{l+m+mi}{0}\PYG{p}{,} \PYG{l+m+mi}{0}\PYG{p}{,} \PYG{l+m+mi}{0}\PYG{p}{]}\PYG{p}{)}  \PYG{c+c1}{\PYGZsh{} [yaw\PYGZus{}reference, sway\PYGZus{}reference, heeling\PYGZus{}reference]}

\PYG{c+c1}{\PYGZsh{} GEOMETRY OF THE KEEL}
\PYG{c+c1}{\PYGZsh{} to estimate heeling moment from keel, does not influence the optimizer.}
\PYG{c+c1}{\PYGZsh{} reminder: the z coord shall be negative (under the water)}
\PYG{n}{center\PYGZus{}of\PYGZus{}lateral\PYGZus{}resistance\PYGZus{}upright} \PYG{o}{=} \PYG{n}{np}\PYG{o}{.}\PYG{n}{array}\PYG{p}{(}\PYG{p}{[}\PYG{l+m+mi}{0}\PYG{p}{,} \PYG{l+m+mi}{0}\PYG{p}{,} \PYG{o}{\PYGZhy{}}\PYG{l+m+mf}{1.0}\PYG{p}{]}\PYG{p}{)}  \PYG{c+c1}{\PYGZsh{} [m] the coordinates for a yacht }
\end{sphinxVerbatim}


\subsection{Run script}
\label{\detokenize{chapters/usage:run-script}}
\sphinxAtStartPar
To run script with variables.py located in working directory:

\begin{sphinxVerbatim}[commandchars=\\\{\}]
\PYG{n}{pySailingVLM}
\end{sphinxVerbatim}

\sphinxAtStartPar
If variables.py is located in different directory, you must sepecify its location by providing additional option for pySailingVLM script:

\begin{sphinxVerbatim}[commandchars=\\\{\}]
\PYG{n}{pySailingVLM} \PYG{o}{\PYGZhy{}}\PYG{o}{\PYGZhy{}}\PYG{n}{dvars} \PYG{n}{path\PYGZus{}to\PYGZus{}foler\PYGZus{}with\PYGZus{}variables}\PYG{o}{.}\PYG{n}{py}
\end{sphinxVerbatim}

\sphinxAtStartPar
More information is available inside script help:

\begin{sphinxVerbatim}[commandchars=\\\{\}]
\PYG{n}{pySailingVLM} \PYG{o}{\PYGZhy{}}\PYG{o}{\PYGZhy{}}\PYG{n}{help}
\end{sphinxVerbatim}


\subsection{Output}
\label{\detokenize{chapters/usage:output}}
\sphinxAtStartPar
pySailingVLM produces output files: data in xlsx file, matplotlib figures and pressure coefficient colormap in html extention (interactive plot). It is saved in the location specified in varaibles.py.


\section{Run from python}
\label{\detokenize{chapters/usage:run-from-python}}
\begin{sphinxVerbatim}[commandchars=\\\{\}]
\PYG{k+kn}{from} \PYG{n+nn}{sailing\PYGZus{}vlm}\PYG{n+nn}{.}\PYG{n+nn}{runner} \PYG{k+kn}{import} \PYG{n}{aircraft} \PYG{k}{as} \PYG{n}{a}
\PYG{k+kn}{import} \PYG{n+nn}{numpy} \PYG{k}{as} \PYG{n+nn}{np}
\PYG{n}{chord\PYGZus{}length} \PYG{o}{=} \PYG{l+m+mf}{1.0}             
\PYG{n}{half\PYGZus{}wing\PYGZus{}span} \PYG{o}{=} \PYG{l+m+mf}{5.0} 
\PYG{n}{AoA\PYGZus{}deg} \PYG{o}{=} \PYG{l+m+mf}{10.}
\PYG{n}{sweep\PYGZus{}angle\PYGZus{}deg} \PYG{o}{=} \PYG{l+m+mf}{0.}
\PYG{n}{tws\PYGZus{}ref} \PYG{o}{=} \PYG{l+m+mf}{1.0}
\PYG{n}{V} \PYG{o}{=} \PYG{n}{np}\PYG{o}{.}\PYG{n}{array}\PYG{p}{(}\PYG{p}{[}\PYG{n}{tws\PYGZus{}ref}\PYG{p}{,} \PYG{l+m+mf}{0.}\PYG{p}{,} \PYG{l+m+mf}{0.}\PYG{p}{]}\PYG{p}{)}
\PYG{n}{rho} \PYG{o}{=} \PYG{l+m+mf}{1.225}
\PYG{n}{gamma\PYGZus{}orientation} \PYG{o}{=} \PYG{o}{\PYGZhy{}}\PYG{l+m+mf}{1.0}
\PYG{n}{n\PYGZus{}spanwise} \PYG{o}{=} \PYG{l+m+mi}{32}  \PYG{c+c1}{\PYGZsh{} No of control points (above the water) per sail, recommended: 50}
\PYG{n}{n\PYGZus{}chordwise} \PYG{o}{=} \PYG{l+m+mi}{8} \PYG{c+c1}{\PYGZsh{}?}
\PYG{n}{b} \PYG{o}{=} \PYG{n}{a}\PYG{o}{.}\PYG{n}{Aircraft}\PYG{p}{(}\PYG{n}{chord\PYGZus{}length}\PYG{p}{,} \PYG{n}{half\PYGZus{}wing\PYGZus{}span}\PYG{p}{,} \PYG{n}{AoA\PYGZus{}deg}\PYG{p}{,} \PYG{n}{n\PYGZus{}spanwise}\PYG{p}{,} \PYG{n}{n\PYGZus{}chordwise}\PYG{p}{,} \PYG{n}{V}\PYG{p}{,} \PYG{n}{rho}\PYG{p}{,} \PYG{n}{gamma\PYGZus{}orientation}\PYG{p}{)}
\PYG{n}{b}\PYG{o}{.}\PYG{n}{Cxyz}
\PYG{o}{\PYGZgt{}\PYGZgt{}}\PYG{o}{\PYGZgt{}} \PYG{p}{(}\PYG{l+m+mf}{0.023472780216173314}\PYG{p}{,} \PYG{l+m+mf}{0.8546846987984326}\PYG{p}{,} \PYG{l+m+mf}{0.0}\PYG{p}{)} 
\end{sphinxVerbatim}


\section{jupyter notebook}
\label{\detokenize{chapters/usage:jupyter-notebook}}
\sphinxAtStartPar
Patrz dołącznoy plik notebooka.

\sphinxstepscope


\chapter{Examples}
\label{\detokenize{chapters/examples:examples}}\label{\detokenize{chapters/examples::doc}}
\sphinxstepscope


\section{Example 1}
\label{\detokenize{chapters/example1:example-1}}\label{\detokenize{chapters/example1::doc}}
\sphinxstepscope


\section{Example 2}
\label{\detokenize{chapters/example2:example-2}}\label{\detokenize{chapters/example2::doc}}
\sphinxstepscope


\chapter{Validation}
\label{\detokenize{chapters/validation:validation}}\label{\detokenize{chapters/validation::doc}}
\sphinxstepscope


\section{Flat\_plate}
\label{\detokenize{chapters/flat_plate:flat-plate}}\label{\detokenize{chapters/flat_plate::doc}}
\sphinxstepscope


\section{Sweep}
\label{\detokenize{chapters/sweep:sweep}}\label{\detokenize{chapters/sweep::doc}}
\sphinxstepscope


\chapter{Content with notebooks}
\label{\detokenize{chapters/notebooks:content-with-notebooks}}\label{\detokenize{chapters/notebooks::doc}}
\sphinxAtStartPar
You can also create content with Jupyter Notebooks. This means that you can include
code blocks and their outputs in your book.

\begin{sphinxuseclass}{cell}\begin{sphinxVerbatimInput}

\begin{sphinxuseclass}{cell_input}
\begin{sphinxVerbatim}[commandchars=\\\{\}]
\PYG{c+c1}{\PYGZsh{} varaibles.py for jupyter}
\PYG{k+kn}{import} \PYG{n+nn}{os}
\PYG{k+kn}{import} \PYG{n+nn}{numpy} \PYG{k}{as} \PYG{n+nn}{np}
\PYG{k+kn}{import} \PYG{n+nn}{time}

\PYG{c+c1}{\PYGZsh{} OUTPUT DIR}
\PYG{n}{case\PYGZus{}dir} \PYG{o}{=} \PYG{n}{os}\PYG{o}{.}\PYG{n}{path}\PYG{o}{.}\PYG{n}{abspath}\PYG{p}{(}\PYG{l+s+s1}{\PYGZsq{}}\PYG{l+s+s1}{\PYGZsq{}}\PYG{p}{)} \PYG{c+c1}{\PYGZsh{} current directory}
\PYG{n}{time\PYGZus{}stamp} \PYG{o}{=} \PYG{n}{time}\PYG{o}{.}\PYG{n}{strftime}\PYG{p}{(}\PYG{l+s+s2}{\PYGZdq{}}\PYG{l+s+s2}{\PYGZpc{}}\PYG{l+s+s2}{Y\PYGZhy{}}\PYG{l+s+s2}{\PYGZpc{}}\PYG{l+s+s2}{m\PYGZhy{}}\PYG{l+s+si}{\PYGZpc{}d}\PYG{l+s+s2}{\PYGZus{}}\PYG{l+s+s2}{\PYGZpc{}}\PYG{l+s+s2}{Hh}\PYG{l+s+s2}{\PYGZpc{}}\PYG{l+s+s2}{Mm}\PYG{l+s+s2}{\PYGZpc{}}\PYG{l+s+s2}{Ss}\PYG{l+s+s2}{\PYGZdq{}}\PYG{p}{)}
\PYG{n}{output\PYGZus{}dir\PYGZus{}name} \PYG{o}{=} \PYG{n}{os}\PYG{o}{.}\PYG{n}{path}\PYG{o}{.}\PYG{n}{join}\PYG{p}{(}\PYG{l+s+s2}{\PYGZdq{}}\PYG{l+s+s2}{results\PYGZus{}example\PYGZus{}jib\PYGZus{}and\PYGZus{}mainsail\PYGZus{}vlm}\PYG{l+s+s2}{\PYGZdq{}}\PYG{p}{,} \PYG{n}{time\PYGZus{}stamp}\PYG{p}{)}
\PYG{n}{file\PYGZus{}name} \PYG{o}{=} \PYG{l+s+s1}{\PYGZsq{}}\PYG{l+s+s1}{my\PYGZus{}fancy\PYGZus{}results}\PYG{l+s+s1}{\PYGZsq{}} \PYG{c+c1}{\PYGZsh{} name of xlsx excel file}

\PYG{c+c1}{\PYGZsh{} SOLVER SETTINGS}
\PYG{n}{n\PYGZus{}spanwise} \PYG{o}{=} \PYG{l+m+mi}{4}  \PYG{c+c1}{\PYGZsh{} No of control points (above the water) per sail, recommended: 50}
\PYG{n}{n\PYGZus{}chordwise} \PYG{o}{=} \PYG{l+m+mi}{1} \PYG{c+c1}{\PYGZsh{} No of control points (above the water) per sail, recommended: 50}
\PYG{n}{interpolation\PYGZus{}type} \PYG{o}{=} \PYG{l+s+s2}{\PYGZdq{}}\PYG{l+s+s2}{linear}\PYG{l+s+s2}{\PYGZdq{}}  \PYG{c+c1}{\PYGZsh{} either \PYGZdq{}spline\PYGZdq{} or \PYGZdq{}linear\PYGZdq{}}
\PYG{n}{LLT\PYGZus{}twist} \PYG{o}{=} \PYG{l+s+s2}{\PYGZdq{}}\PYG{l+s+s2}{real\PYGZus{}twist}\PYG{l+s+s2}{\PYGZdq{}}  \PYG{c+c1}{\PYGZsh{} defines how the Lifting Line discretize the sail twist.}
\PYG{c+c1}{\PYGZsh{} It can be \PYGZdq{}sheeting\PYGZus{}angle\PYGZus{}const\PYGZdq{} or \PYGZdq{}average\PYGZus{}const\PYGZdq{} or \PYGZdq{}real\PYGZus{}twist\PYGZdq{}}

\PYG{c+c1}{\PYGZsh{} SAILING CONDITIONS}
\PYG{n}{leeway\PYGZus{}deg} \PYG{o}{=} \PYG{l+m+mf}{0.}    \PYG{c+c1}{\PYGZsh{} [deg]}
\PYG{n}{heel\PYGZus{}deg} \PYG{o}{=} \PYG{l+m+mf}{0.}     \PYG{c+c1}{\PYGZsh{} [deg]}
\PYG{n}{SOG\PYGZus{}yacht} \PYG{o}{=} \PYG{l+m+mf}{0.}   \PYG{c+c1}{\PYGZsh{} [m/s] yacht speed \PYGZhy{} speed over ground (leeway is a separate variable)}
\PYG{n}{tws\PYGZus{}ref} \PYG{o}{=} \PYG{l+m+mf}{1.}     \PYG{c+c1}{\PYGZsh{} [m/s] true wind speed}
\PYG{n}{alpha\PYGZus{}true\PYGZus{}wind\PYGZus{}deg} \PYG{o}{=} \PYG{l+m+mi}{0}\PYG{c+c1}{\PYGZsh{}10.   \PYGZsh{} [deg] true wind angle (with reference to course over ground) =\PYGZgt{} Course Wind Angle to the boat track = true wind angle to centerline + Leeway}
\PYG{n}{reference\PYGZus{}water\PYGZus{}level\PYGZus{}for\PYGZus{}wind\PYGZus{}profile} \PYG{o}{=} \PYG{l+m+mf}{0.}  \PYG{c+c1}{\PYGZsh{} [m] this is an attempt to mimick the deck effect}
\PYG{c+c1}{\PYGZsh{} by lowering the sheer\PYGZus{}above\PYGZus{}waterline}
\PYG{c+c1}{\PYGZsh{} while keeping the wind profile as in original geometry}
\PYG{c+c1}{\PYGZsh{} this shall be negative (H = sail\PYGZus{}ctrl\PYGZus{}point \PYGZhy{} water\PYGZus{}level)}
\PYG{n}{wind\PYGZus{}exp\PYGZus{}coeff} \PYG{o}{=} \PYG{l+m+mf}{0.}  \PYG{c+c1}{\PYGZsh{} [\PYGZhy{}] coefficient to determine the exponential wind profile}
\PYG{n}{wind\PYGZus{}reference\PYGZus{}measurment\PYGZus{}height} \PYG{o}{=} \PYG{l+m+mf}{10.}  \PYG{c+c1}{\PYGZsh{} [m] reference height for exponential wind profile}
\PYG{n}{rho} \PYG{o}{=} \PYG{l+m+mf}{1.225}  \PYG{c+c1}{\PYGZsh{} air density [kg/m3]}
\PYG{n}{wind\PYGZus{}profile} \PYG{o}{=} \PYG{l+s+s1}{\PYGZsq{}}\PYG{l+s+s1}{flat}\PYG{l+s+s1}{\PYGZsq{}} \PYG{c+c1}{\PYGZsh{} allowed: \PYGZsq{}exponential\PYGZsq{} or \PYGZsq{}flat\PYGZsq{} or \PYGZsq{}logarithmic\PYGZsq{}}
\PYG{n}{roughness} \PYG{o}{=} \PYG{l+m+mf}{0.05} \PYG{c+c1}{\PYGZsh{} for logarithmic profile only}

\PYG{c+c1}{\PYGZsh{} GEOMETRY OF THE RIG}
\PYG{n}{main\PYGZus{}sail\PYGZus{}luff} \PYG{o}{=} \PYG{l+m+mf}{0.5}\PYG{c+c1}{\PYGZsh{}10. \PYGZsh{} 12.4  \PYGZsh{} [m]}
\PYG{n}{jib\PYGZus{}luff} \PYG{o}{=} \PYG{l+m+mf}{10.0}  \PYG{c+c1}{\PYGZsh{} [m]}
\PYG{n}{foretriangle\PYGZus{}height} \PYG{o}{=} \PYG{l+m+mf}{11.50}  \PYG{c+c1}{\PYGZsh{} [m]}
\PYG{n}{foretriangle\PYGZus{}base} \PYG{o}{=} \PYG{l+m+mf}{3.90}  \PYG{c+c1}{\PYGZsh{} [m]}
\PYG{n}{sheer\PYGZus{}above\PYGZus{}waterline} \PYG{o}{=} \PYG{l+m+mf}{0.}  \PYG{c+c1}{\PYGZsh{} [m]}
\PYG{n}{boom\PYGZus{}above\PYGZus{}sheer} \PYG{o}{=} \PYG{l+m+mf}{0.} \PYG{c+c1}{\PYGZsh{}1.3  \PYGZsh{} [m]}
\PYG{n}{rake\PYGZus{}deg} \PYG{o}{=} \PYG{l+m+mi}{45}\PYG{o}{+}\PYG{l+m+mf}{90.}  \PYG{c+c1}{\PYGZsh{} rake angle [deg]}
\PYG{n}{mast\PYGZus{}LOA} \PYG{o}{=} \PYG{l+m+mf}{0.0}  \PYG{c+c1}{\PYGZsh{} [m]}

\PYG{c+c1}{\PYGZsh{} INPUT \PYGZhy{} GEOMETRY OF THE SAIL}
\PYG{n}{sails\PYGZus{}def} \PYG{o}{=} \PYG{l+s+s1}{\PYGZsq{}}\PYG{l+s+s1}{main}\PYG{l+s+s1}{\PYGZsq{}} \PYG{c+c1}{\PYGZsh{} definition of sail set, possible: \PYGZsq{}jib\PYGZsq{} or \PYGZsq{}main\PYGZsq{} or \PYGZsq{}jib\PYGZus{}and\PYGZus{}main\PYGZsq{}}
\PYG{c+c1}{\PYGZsh{} sails\PYGZus{}def = \PYGZsq{}main\PYGZsq{}}
\PYG{n}{main\PYGZus{}sail\PYGZus{}girths} \PYG{o}{=} \PYG{n}{np}\PYG{o}{.}\PYG{n}{array}\PYG{p}{(}\PYG{p}{[}\PYG{l+m+mf}{0.00}\PYG{p}{,} \PYG{l+m+mf}{1.}\PYG{o}{/}\PYG{l+m+mi}{8}\PYG{p}{,} \PYG{l+m+mf}{1.}\PYG{o}{/}\PYG{l+m+mi}{4}\PYG{p}{,} \PYG{l+m+mf}{1.}\PYG{o}{/}\PYG{l+m+mi}{2}\PYG{p}{,} \PYG{l+m+mf}{3.}\PYG{o}{/}\PYG{l+m+mi}{4}\PYG{p}{,} \PYG{l+m+mf}{7.}\PYG{o}{/}\PYG{l+m+mi}{8}\PYG{p}{,} \PYG{l+m+mf}{1.00}\PYG{p}{]}\PYG{p}{)}
\PYG{n}{main\PYGZus{}sail\PYGZus{}chords} \PYG{o}{=} \PYG{n}{np}\PYG{o}{.}\PYG{n}{array}\PYG{p}{(}\PYG{p}{[}\PYG{l+m+mf}{0.2}\PYG{p}{]}\PYG{o}{*} \PYG{n+nb}{len}\PYG{p}{(}\PYG{n}{main\PYGZus{}sail\PYGZus{}girths}\PYG{p}{)}\PYG{p}{)} \PYG{c+c1}{\PYGZsh{} np.array([4.00, 3.82, 3.64, 3.20, 2.64, 2.32, 2.00])}
\PYG{n}{main\PYGZus{}sail\PYGZus{}centerline\PYGZus{}twist\PYGZus{}deg} \PYG{o}{=} \PYG{o}{\PYGZhy{}}\PYG{l+m+mi}{2} \PYG{o}{+} \PYG{l+m+mi}{0}\PYG{o}{*} \PYG{n}{main\PYGZus{}sail\PYGZus{}girths} \PYG{c+c1}{\PYGZsh{} 10 + 12. * main\PYGZus{}sail\PYGZus{}girths  \PYGZsh{}}

\PYG{c+c1}{\PYGZsh{} First digit describing maximum camber as percentage of the chord.}
\PYG{c+c1}{\PYGZsh{} Second digit describing the distance of maximum camber from the airfoil leading edge in tenths of the chord.}
\PYG{n}{jib\PYGZus{}sail\PYGZus{}camber}\PYG{o}{=} \PYG{l+m+mi}{0}\PYG{o}{*}\PYG{n}{np}\PYG{o}{.}\PYG{n}{array}\PYG{p}{(}\PYG{p}{[}\PYG{l+m+mf}{0.01}\PYG{p}{,} \PYG{l+m+mf}{0.01}\PYG{p}{,} \PYG{l+m+mf}{0.01}\PYG{p}{,} \PYG{l+m+mf}{0.01}\PYG{p}{,} \PYG{l+m+mf}{0.01}\PYG{p}{]}\PYG{p}{)}
\PYG{n}{jib\PYGZus{}sail\PYGZus{}camber\PYGZus{}distance\PYGZus{}from\PYGZus{}luff} \PYG{o}{=} \PYG{n}{np}\PYG{o}{.}\PYG{n}{array}\PYG{p}{(}\PYG{p}{[}\PYG{l+m+mf}{0.5}\PYG{p}{,} \PYG{l+m+mf}{0.5}\PYG{p}{,} \PYG{l+m+mf}{0.5}\PYG{p}{,} \PYG{l+m+mf}{0.5}\PYG{p}{,} \PYG{l+m+mf}{0.5}\PYG{p}{]}\PYG{p}{)} \PYG{c+c1}{\PYGZsh{} starting from leading edge}
\PYG{n}{main\PYGZus{}sail\PYGZus{}camber}\PYG{o}{=} \PYG{l+m+mi}{0}\PYG{o}{*}\PYG{n}{np}\PYG{o}{.}\PYG{n}{array}\PYG{p}{(}\PYG{p}{[}\PYG{l+m+mf}{0.01}\PYG{p}{,} \PYG{l+m+mf}{0.01}\PYG{p}{,} \PYG{l+m+mf}{0.01}\PYG{p}{,} \PYG{l+m+mf}{0.01}\PYG{p}{,} \PYG{l+m+mf}{0.01}\PYG{p}{,} \PYG{l+m+mf}{0.01}\PYG{p}{,} \PYG{l+m+mf}{0.01}\PYG{p}{]}\PYG{p}{)}
\PYG{n}{main\PYGZus{}sail\PYGZus{}camber\PYGZus{}distance\PYGZus{}from\PYGZus{}luff} \PYG{o}{=} \PYG{n}{np}\PYG{o}{.}\PYG{n}{array}\PYG{p}{(}\PYG{p}{[}\PYG{l+m+mf}{0.5}\PYG{p}{,} \PYG{l+m+mf}{0.5}\PYG{p}{,} \PYG{l+m+mf}{0.5}\PYG{p}{,} \PYG{l+m+mf}{0.5}\PYG{p}{,} \PYG{l+m+mf}{0.5}\PYG{p}{,} \PYG{l+m+mf}{0.5}\PYG{p}{,} \PYG{l+m+mf}{0.5}\PYG{p}{]}\PYG{p}{)} \PYG{c+c1}{\PYGZsh{} starting from leading edge}

\PYG{n}{jib\PYGZus{}girths} \PYG{o}{=} \PYG{n}{np}\PYG{o}{.}\PYG{n}{array}\PYG{p}{(}\PYG{p}{[}\PYG{l+m+mf}{0.00}\PYG{p}{,} \PYG{l+m+mf}{1.}\PYG{o}{/}\PYG{l+m+mi}{4}\PYG{p}{,} \PYG{l+m+mf}{1.}\PYG{o}{/}\PYG{l+m+mi}{2}\PYG{p}{,} \PYG{l+m+mf}{3.}\PYG{o}{/}\PYG{l+m+mi}{4}\PYG{p}{,} \PYG{l+m+mf}{1.00}\PYG{p}{]}\PYG{p}{)}
\PYG{n}{jib\PYGZus{}chords} \PYG{o}{=} \PYG{l+m+mf}{1E\PYGZhy{}12}\PYG{o}{*} \PYG{n}{np}\PYG{o}{.}\PYG{n}{array}\PYG{p}{(}\PYG{p}{[}\PYG{l+m+mf}{3.80}\PYG{p}{,} \PYG{l+m+mf}{2.98}\PYG{p}{,} \PYG{l+m+mf}{2.15}\PYG{p}{,} \PYG{l+m+mf}{1.33}\PYG{p}{,} \PYG{l+m+mf}{0.5}\PYG{p}{]}\PYG{p}{)}
\PYG{n}{jib\PYGZus{}centerline\PYGZus{}twist\PYGZus{}deg} \PYG{o}{=}  \PYG{l+m+mi}{0}\PYG{o}{*}\PYG{p}{(}\PYG{l+m+mi}{10}\PYG{o}{+}\PYG{l+m+mi}{5}\PYG{p}{)}  \PYG{o}{+} \PYG{l+m+mi}{0}\PYG{o}{*}\PYG{l+m+mf}{15.} \PYG{o}{*} \PYG{n}{jib\PYGZus{}girths} \PYG{c+c1}{\PYGZsh{} (10+5)  + 15. * jib\PYGZus{}girths \PYGZsh{} }

\PYG{c+c1}{\PYGZsh{} REFERENCE CSYS}
\PYG{c+c1}{\PYGZsh{} The origin of the default CSYS is located @ waterline level and aft face of the mast}
\PYG{c+c1}{\PYGZsh{} The positive x\PYGZhy{}coord: towards stern}
\PYG{c+c1}{\PYGZsh{} The positive y\PYGZhy{}coord: towards leeward side}
\PYG{c+c1}{\PYGZsh{} The positive z\PYGZhy{}coord: above the water}
\PYG{c+c1}{\PYGZsh{} To shift the default CSYS, adjust the \PYGZsq{}reference\PYGZus{}level\PYGZus{}for\PYGZus{}moments\PYGZsq{} variable.}
\PYG{c+c1}{\PYGZsh{} Shifted CSYS = original + reference\PYGZus{}level\PYGZus{}for\PYGZus{}moments}
\PYG{c+c1}{\PYGZsh{} As a results the moments will be calculated around the new origin.}

\PYG{c+c1}{\PYGZsh{} yaw\PYGZus{}reference [m] \PYGZhy{} distance from the aft of the mast towards stern, at which the yawing moment is calculated.}
\PYG{c+c1}{\PYGZsh{} sway\PYGZus{}reference [m] \PYGZhy{} distance from the aft of the mast towards leeward side. 0 for symmetric yachts ;)}
\PYG{c+c1}{\PYGZsh{} heeling\PYGZus{}reference [m] \PYGZhy{} distance from the water level,  at which the heeling moment is calculated.}
\PYG{n}{reference\PYGZus{}level\PYGZus{}for\PYGZus{}moments} \PYG{o}{=} \PYG{n}{np}\PYG{o}{.}\PYG{n}{array}\PYG{p}{(}\PYG{p}{[}\PYG{l+m+mi}{0}\PYG{p}{,} \PYG{l+m+mi}{0}\PYG{p}{,} \PYG{l+m+mi}{0}\PYG{p}{]}\PYG{p}{)}  \PYG{c+c1}{\PYGZsh{} [yaw\PYGZus{}reference, sway\PYGZus{}reference, heeling\PYGZus{}reference]}

\PYG{c+c1}{\PYGZsh{} GEOMETRY OF THE KEEL}
\PYG{c+c1}{\PYGZsh{} to estimate heeling moment from keel, does not influence the optimizer.}
\PYG{c+c1}{\PYGZsh{} reminder: the z coord shall be negative (under the water)}
\PYG{n}{center\PYGZus{}of\PYGZus{}lateral\PYGZus{}resistance\PYGZus{}upright} \PYG{o}{=} \PYG{n}{np}\PYG{o}{.}\PYG{n}{array}\PYG{p}{(}\PYG{p}{[}\PYG{l+m+mi}{0}\PYG{p}{,} \PYG{l+m+mi}{0}\PYG{p}{,} \PYG{o}{\PYGZhy{}}\PYG{l+m+mf}{1.0}\PYG{p}{]}\PYG{p}{)}  \PYG{c+c1}{\PYGZsh{} [m] the coordinates for a yacht standing in upright position}
\end{sphinxVerbatim}

\end{sphinxuseclass}\end{sphinxVerbatimInput}

\end{sphinxuseclass}
\begin{sphinxuseclass}{cell}\begin{sphinxVerbatimInput}

\begin{sphinxuseclass}{cell_input}
\begin{sphinxVerbatim}[commandchars=\\\{\}]
\PYG{k+kn}{from} \PYG{n+nn}{pySailingVLM}\PYG{n+nn}{.}\PYG{n+nn}{rotations}\PYG{n+nn}{.}\PYG{n+nn}{csys\PYGZus{}transformations} \PYG{k+kn}{import} \PYG{n}{CSYS\PYGZus{}transformations}
\PYG{k+kn}{from} \PYG{n+nn}{pySailingVLM}\PYG{n+nn}{.}\PYG{n+nn}{yacht\PYGZus{}geometry}\PYG{n+nn}{.}\PYG{n+nn}{hull\PYGZus{}geometry} \PYG{k+kn}{import} \PYG{n}{HullGeometry}
\PYG{k+kn}{from} \PYG{n+nn}{pySailingVLM}\PYG{n+nn}{.}\PYG{n+nn}{results}\PYG{n+nn}{.}\PYG{n+nn}{save\PYGZus{}utils} \PYG{k+kn}{import} \PYG{n}{save\PYGZus{}results\PYGZus{}to\PYGZus{}file}
\PYG{k+kn}{from} \PYG{n+nn}{pySailingVLM}\PYG{n+nn}{.}\PYG{n+nn}{solver}\PYG{n+nn}{.}\PYG{n+nn}{panels\PYGZus{}plotter} \PYG{k+kn}{import} \PYG{n}{display\PYGZus{}panels\PYGZus{}xyz\PYGZus{}and\PYGZus{}winds}
\PYG{k+kn}{from} \PYG{n+nn}{pySailingVLM}\PYG{n+nn}{.}\PYG{n+nn}{results}\PYG{n+nn}{.}\PYG{n+nn}{inviscid\PYGZus{}flow} \PYG{k+kn}{import} \PYG{n}{InviscidFlowResults}
\PYG{k+kn}{from} \PYG{n+nn}{pySailingVLM}\PYG{n+nn}{.}\PYG{n+nn}{solver}\PYG{n+nn}{.}\PYG{n+nn}{coefs} \PYG{k+kn}{import} \PYG{n}{get\PYGZus{}vlm\PYGZus{}Cxyz}
\PYG{k+kn}{from} \PYG{n+nn}{pySailingVLM}\PYG{n+nn}{.}\PYG{n+nn}{solver}\PYG{n+nn}{.}\PYG{n+nn}{vlm} \PYG{k+kn}{import} \PYG{n}{Vlm}
\PYG{k+kn}{from} \PYG{n+nn}{pySailingVLM}\PYG{n+nn}{.}\PYG{n+nn}{runner}\PYG{n+nn}{.}\PYG{n+nn}{sail} \PYG{k+kn}{import} \PYG{n}{Wind}\PYG{p}{,} \PYG{n}{Sail}

\PYG{k+kn}{from} \PYG{n+nn}{pySailingVLM}\PYG{n+nn}{.}\PYG{n+nn}{solver}\PYG{n+nn}{.}\PYG{n+nn}{panels\PYGZus{}plotter} \PYG{k+kn}{import} \PYG{n}{plot\PYGZus{}cp}
\end{sphinxVerbatim}

\end{sphinxuseclass}\end{sphinxVerbatimInput}

\end{sphinxuseclass}
\begin{sphinxuseclass}{cell}\begin{sphinxVerbatimInput}

\begin{sphinxuseclass}{cell_input}
\begin{sphinxVerbatim}[commandchars=\\\{\}]
\PYG{n}{csys\PYGZus{}transformations} \PYG{o}{=} \PYG{n}{CSYS\PYGZus{}transformations}\PYG{p}{(}
    \PYG{n}{heel\PYGZus{}deg}\PYG{p}{,} \PYG{n}{leeway\PYGZus{}deg}\PYG{p}{,}
    \PYG{n}{v\PYGZus{}from\PYGZus{}original\PYGZus{}xyz\PYGZus{}2\PYGZus{}reference\PYGZus{}csys\PYGZus{}xyz}\PYG{o}{=}\PYG{n}{reference\PYGZus{}level\PYGZus{}for\PYGZus{}moments}\PYG{p}{)}

\PYG{n}{w} \PYG{o}{=} \PYG{n}{Wind}\PYG{p}{(}\PYG{n}{alpha\PYGZus{}true\PYGZus{}wind\PYGZus{}deg}\PYG{p}{,} \PYG{n}{tws\PYGZus{}ref}\PYG{p}{,} \PYG{n}{SOG\PYGZus{}yacht}\PYG{p}{,} \PYG{n}{wind\PYGZus{}exp\PYGZus{}coeff}\PYG{p}{,} \PYG{n}{wind\PYGZus{}reference\PYGZus{}measurment\PYGZus{}height}\PYG{p}{,} 
         \PYG{n}{reference\PYGZus{}water\PYGZus{}level\PYGZus{}for\PYGZus{}wind\PYGZus{}profile}\PYG{p}{,} \PYG{n}{roughness}\PYG{p}{,} \PYG{n}{wind\PYGZus{}profile}\PYG{p}{)}
\PYG{n}{w\PYGZus{}profile} \PYG{o}{=} \PYG{n}{w}\PYG{o}{.}\PYG{n}{profile}

\PYG{n}{s} \PYG{o}{=} \PYG{n}{Sail}\PYG{p}{(}\PYG{n}{n\PYGZus{}spanwise}\PYG{p}{,} \PYG{n}{n\PYGZus{}chordwise}\PYG{p}{,} \PYG{n}{csys\PYGZus{}transformations}\PYG{p}{,} \PYG{n}{sheer\PYGZus{}above\PYGZus{}waterline}\PYG{p}{,} \PYG{n}{rake\PYGZus{}deg}\PYG{p}{,} \PYG{n}{boom\PYGZus{}above\PYGZus{}sheer}\PYG{p}{,} 
        \PYG{n}{mast\PYGZus{}LOA}\PYG{p}{,} \PYG{n}{main\PYGZus{}sail\PYGZus{}luff}\PYG{p}{,} \PYG{n}{main\PYGZus{}sail\PYGZus{}girths}\PYG{p}{,} \PYG{n}{main\PYGZus{}sail\PYGZus{}chords}\PYG{p}{,} \PYG{n}{main\PYGZus{}sail\PYGZus{}centerline\PYGZus{}twist\PYGZus{}deg}\PYG{p}{,} 
        \PYG{n}{main\PYGZus{}sail\PYGZus{}camber}\PYG{p}{,} \PYG{n}{main\PYGZus{}sail\PYGZus{}camber\PYGZus{}distance\PYGZus{}from\PYGZus{}luff}\PYG{p}{,} \PYG{n}{foretriangle\PYGZus{}base}\PYG{p}{,} \PYG{n}{foretriangle\PYGZus{}height}\PYG{p}{,} 
        \PYG{n}{jib\PYGZus{}luff}\PYG{p}{,} \PYG{n}{jib\PYGZus{}girths}\PYG{p}{,} \PYG{n}{jib\PYGZus{}chords}\PYG{p}{,} \PYG{n}{jib\PYGZus{}centerline\PYGZus{}twist\PYGZus{}deg}\PYG{p}{,} \PYG{n}{jib\PYGZus{}sail\PYGZus{}camber}\PYG{p}{,} 
        \PYG{n}{jib\PYGZus{}sail\PYGZus{}camber\PYGZus{}distance\PYGZus{}from\PYGZus{}luff}\PYG{p}{,} \PYG{n}{sails\PYGZus{}def}\PYG{p}{,} \PYG{n}{LLT\PYGZus{}twist}\PYG{p}{,} \PYG{n}{interpolation\PYGZus{}type}\PYG{p}{)}
\PYG{n}{sail\PYGZus{}set} \PYG{o}{=} \PYG{n}{s}\PYG{o}{.}\PYG{n}{sail\PYGZus{}set}
\PYG{n}{hull} \PYG{o}{=} \PYG{n}{HullGeometry}\PYG{p}{(}\PYG{n}{sheer\PYGZus{}above\PYGZus{}waterline}\PYG{p}{,} \PYG{n}{foretriangle\PYGZus{}base}\PYG{p}{,} \PYG{n}{csys\PYGZus{}transformations}\PYG{p}{,} \PYG{n}{center\PYGZus{}of\PYGZus{}lateral\PYGZus{}resistance\PYGZus{}upright}\PYG{p}{)}
\PYG{n}{myvlm} \PYG{o}{=} \PYG{n}{Vlm}\PYG{p}{(}\PYG{n}{sail\PYGZus{}set}\PYG{o}{.}\PYG{n}{panels}\PYG{p}{,} \PYG{n}{n\PYGZus{}chordwise}\PYG{p}{,} \PYG{n}{n\PYGZus{}spanwise}\PYG{p}{,} \PYG{n}{rho}\PYG{p}{,} \PYG{n}{w\PYGZus{}profile}\PYG{p}{,} \PYG{n}{sail\PYGZus{}set}\PYG{o}{.}\PYG{n}{trailing\PYGZus{}edge\PYGZus{}info}\PYG{p}{,} \PYG{n}{sail\PYGZus{}set}\PYG{o}{.}\PYG{n}{leading\PYGZus{}edge\PYGZus{}info}\PYG{p}{)}

\PYG{n}{inviscid\PYGZus{}flow\PYGZus{}results} \PYG{o}{=} \PYG{n}{InviscidFlowResults}\PYG{p}{(}\PYG{n}{sail\PYGZus{}set}\PYG{p}{,} \PYG{n}{csys\PYGZus{}transformations}\PYG{p}{,} \PYG{n}{myvlm}\PYG{p}{)}

\PYG{n}{inviscid\PYGZus{}flow\PYGZus{}results}\PYG{o}{.}\PYG{n}{estimate\PYGZus{}heeling\PYGZus{}moment\PYGZus{}from\PYGZus{}keel}\PYG{p}{(}\PYG{n}{hull}\PYG{o}{.}\PYG{n}{center\PYGZus{}of\PYGZus{}lateral\PYGZus{}resistance}\PYG{p}{)}
\end{sphinxVerbatim}

\end{sphinxuseclass}\end{sphinxVerbatimInput}
\begin{sphinxVerbatimOutput}

\begin{sphinxuseclass}{cell_output}
\begin{sphinxVerbatim}[commandchars=\\\{\}]
Applying initial\PYGZus{}sail\PYGZus{}twist\PYGZus{}deg to main\PYGZus{}sail \PYGZhy{}  Lifting Line, mode: real\PYGZus{}twist
\end{sphinxVerbatim}

\end{sphinxuseclass}\end{sphinxVerbatimOutput}

\end{sphinxuseclass}
\begin{sphinxuseclass}{cell}\begin{sphinxVerbatimInput}

\begin{sphinxuseclass}{cell_input}
\begin{sphinxVerbatim}[commandchars=\\\{\}]
\PYG{c+c1}{\PYGZsh{} TODO: by matplotlib byl interaktywny w notebooku}
\PYG{n+nb}{print}\PYG{p}{(}\PYG{l+s+s2}{\PYGZdq{}}\PYG{l+s+s2}{Preparing visualization.}\PYG{l+s+s2}{\PYGZdq{}}\PYG{p}{)}   
\PYG{n}{display\PYGZus{}panels\PYGZus{}xyz\PYGZus{}and\PYGZus{}winds}\PYG{p}{(}\PYG{n}{myvlm}\PYG{p}{,} \PYG{n}{inviscid\PYGZus{}flow\PYGZus{}results}\PYG{p}{,} \PYG{n}{myvlm}\PYG{o}{.}\PYG{n}{inlet\PYGZus{}conditions}\PYG{p}{,} \PYG{n}{hull}\PYG{p}{,} \PYG{n}{show\PYGZus{}plot}\PYG{o}{=}\PYG{k+kc}{True}\PYG{p}{)}
\PYG{n}{df\PYGZus{}components}\PYG{p}{,} \PYG{n}{df\PYGZus{}integrals}\PYG{p}{,} \PYG{n}{df\PYGZus{}inlet\PYGZus{}IC} \PYG{o}{=} \PYG{n}{save\PYGZus{}results\PYGZus{}to\PYGZus{}file}\PYG{p}{(}\PYG{n}{myvlm}\PYG{p}{,} \PYG{n}{csys\PYGZus{}transformations}\PYG{p}{,} 
                                \PYG{n}{inviscid\PYGZus{}flow\PYGZus{}results}\PYG{p}{,} \PYG{n}{sail\PYGZus{}set}\PYG{p}{,} \PYG{n}{output\PYGZus{}dir\PYGZus{}name}\PYG{p}{,} \PYG{n}{file\PYGZus{}name}\PYG{p}{)}

\PYG{n+nb}{print}\PYG{p}{(}\PYG{l+s+sa}{f}\PYG{l+s+s2}{\PYGZdq{}}\PYG{l+s+s2}{\PYGZhy{}\PYGZhy{}\PYGZhy{}\PYGZhy{}\PYGZhy{}\PYGZhy{}\PYGZhy{}\PYGZhy{}\PYGZhy{}\PYGZhy{}\PYGZhy{}\PYGZhy{}\PYGZhy{}\PYGZhy{}\PYGZhy{}\PYGZhy{}\PYGZhy{}\PYGZhy{}\PYGZhy{}\PYGZhy{}\PYGZhy{}\PYGZhy{}\PYGZhy{}\PYGZhy{}\PYGZhy{}\PYGZhy{}\PYGZhy{}\PYGZhy{}\PYGZhy{}\PYGZhy{}\PYGZhy{}\PYGZhy{}\PYGZhy{}\PYGZhy{}\PYGZhy{}\PYGZhy{}\PYGZhy{}\PYGZhy{}\PYGZhy{}\PYGZhy{}\PYGZhy{}\PYGZhy{}\PYGZhy{}\PYGZhy{}\PYGZhy{}\PYGZhy{}\PYGZhy{}\PYGZhy{}\PYGZhy{}\PYGZhy{}\PYGZhy{}\PYGZhy{}\PYGZhy{}\PYGZhy{}\PYGZhy{}\PYGZhy{}\PYGZhy{}\PYGZhy{}\PYGZhy{}\PYGZhy{}\PYGZhy{}}\PYG{l+s+s2}{\PYGZdq{}}\PYG{p}{)}
\PYG{n+nb}{print}\PYG{p}{(}\PYG{l+s+sa}{f}\PYG{l+s+s2}{\PYGZdq{}}\PYG{l+s+s2}{Notice:}\PYG{l+s+se}{\PYGZbs{}n}\PYG{l+s+s2}{\PYGZdq{}}
      \PYG{l+s+sa}{f}\PYG{l+s+s2}{\PYGZdq{}}\PYG{l+s+se}{\PYGZbs{}t}\PYG{l+s+s2}{The forces [N] and moments [Nm] are without profile drag.}\PYG{l+s+se}{\PYGZbs{}n}\PYG{l+s+s2}{\PYGZdq{}}
      \PYG{l+s+sa}{f}\PYG{l+s+s2}{\PYGZdq{}}\PYG{l+s+se}{\PYGZbs{}t}\PYG{l+s+s2}{The the \PYGZus{}COG\PYGZus{} CSYS is aligned in the direction of the yacht movement (course over ground).}\PYG{l+s+se}{\PYGZbs{}n}\PYG{l+s+s2}{\PYGZdq{}}
      \PYG{l+s+sa}{f}\PYG{l+s+s2}{\PYGZdq{}}\PYG{l+s+se}{\PYGZbs{}t}\PYG{l+s+s2}{The the \PYGZus{}COW\PYGZus{} CSYS is aligned along the centerline of the yacht (course over water).}\PYG{l+s+se}{\PYGZbs{}n}\PYG{l+s+s2}{\PYGZdq{}}
      \PYG{l+s+sa}{f}\PYG{l+s+s2}{\PYGZdq{}}\PYG{l+s+se}{\PYGZbs{}t}\PYG{l+s+s2}{Number of panels (sail sail\PYGZus{}set with mirror): }\PYG{l+s+si}{\PYGZob{}}\PYG{n}{sail\PYGZus{}set}\PYG{o}{.}\PYG{n}{panels}\PYG{o}{.}\PYG{n}{shape}\PYG{l+s+si}{\PYGZcb{}}\PYG{l+s+s2}{\PYGZdq{}}\PYG{p}{)}

\PYG{n}{df\PYGZus{}integrals}
\end{sphinxVerbatim}

\end{sphinxuseclass}\end{sphinxVerbatimInput}
\begin{sphinxVerbatimOutput}

\begin{sphinxuseclass}{cell_output}
\begin{sphinxVerbatim}[commandchars=\\\{\}]
Preparing visualization.
\end{sphinxVerbatim}

\noindent\sphinxincludegraphics{{4d4112804d927d596592dd57cc574db2b5789b22568159a64a861ea7dbfff808}.png}

\begin{sphinxVerbatim}[commandchars=\\\{\}]
\PYGZhy{}\PYGZhy{}\PYGZhy{}\PYGZhy{}\PYGZhy{}\PYGZhy{}\PYGZhy{}\PYGZhy{}\PYGZhy{}\PYGZhy{}\PYGZhy{}\PYGZhy{}\PYGZhy{}\PYGZhy{}\PYGZhy{}\PYGZhy{}\PYGZhy{}\PYGZhy{}\PYGZhy{}\PYGZhy{}\PYGZhy{}\PYGZhy{}\PYGZhy{}\PYGZhy{}\PYGZhy{}\PYGZhy{}\PYGZhy{}\PYGZhy{}\PYGZhy{}\PYGZhy{}\PYGZhy{}\PYGZhy{}\PYGZhy{}\PYGZhy{}\PYGZhy{}\PYGZhy{}\PYGZhy{}\PYGZhy{}\PYGZhy{}\PYGZhy{}\PYGZhy{}\PYGZhy{}\PYGZhy{}\PYGZhy{}\PYGZhy{}\PYGZhy{}\PYGZhy{}\PYGZhy{}\PYGZhy{}\PYGZhy{}\PYGZhy{}\PYGZhy{}\PYGZhy{}\PYGZhy{}\PYGZhy{}\PYGZhy{}\PYGZhy{}\PYGZhy{}\PYGZhy{}\PYGZhy{}\PYGZhy{}
Notice:
	The forces [N] and moments [Nm] are without profile drag.
	The the \PYGZus{}COG\PYGZus{} CSYS is aligned in the direction of the yacht movement (course over ground).
	The the \PYGZus{}COW\PYGZus{} CSYS is aligned along the centerline of the yacht (course over water).
	Number of panels (sail sail\PYGZus{}set with mirror): (8, 4, 3)
\end{sphinxVerbatim}

\begin{sphinxVerbatim}[commandchars=\\\{\}]
                               Quantity     Value
0               F\PYGZus{}main\PYGZus{}sail\PYGZus{}total\PYGZus{}COG.x  0.000019
1               F\PYGZus{}main\PYGZus{}sail\PYGZus{}total\PYGZus{}COG.y  0.003322
2               F\PYGZus{}main\PYGZus{}sail\PYGZus{}total\PYGZus{}COG.z \PYGZhy{}0.000019
3                   F\PYGZus{}sails\PYGZus{}total\PYGZus{}COG.x  0.000019
4                   F\PYGZus{}sails\PYGZus{}total\PYGZus{}COG.y  0.003322
5                   F\PYGZus{}sails\PYGZus{}total\PYGZus{}COG.z \PYGZhy{}0.000019
6                   F\PYGZus{}sails\PYGZus{}total\PYGZus{}COW.x  0.000019
7                   F\PYGZus{}sails\PYGZus{}total\PYGZus{}COW.y  0.003322
8                   F\PYGZus{}sails\PYGZus{}total\PYGZus{}COW.z \PYGZhy{}0.000019
9                    M\PYGZus{}keel\PYGZus{}total\PYGZus{}COG.x \PYGZhy{}0.003322
10                   M\PYGZus{}keel\PYGZus{}total\PYGZus{}COG.y  0.000019
11                   M\PYGZus{}keel\PYGZus{}total\PYGZus{}COG.z  0.000000
12            M\PYGZus{}keel\PYGZus{}total\PYGZus{}COW.x (heel) \PYGZhy{}0.003322
13           M\PYGZus{}keel\PYGZus{}total\PYGZus{}COW.y (pitch)  0.000019
14   M\PYGZus{}keel\PYGZus{}total\PYGZus{}COW.z (yaw \PYGZhy{} JG sign) \PYGZhy{}0.000000
15             M\PYGZus{}keel\PYGZus{}total\PYGZus{}COW.z (yaw)  0.000000
16              M\PYGZus{}main\PYGZus{}sail\PYGZus{}total\PYGZus{}COG.x \PYGZhy{}0.000567
17              M\PYGZus{}main\PYGZus{}sail\PYGZus{}total\PYGZus{}COG.y  0.000005
18              M\PYGZus{}main\PYGZus{}sail\PYGZus{}total\PYGZus{}COG.z  0.000733
19                  M\PYGZus{}sails\PYGZus{}total\PYGZus{}COG.x \PYGZhy{}0.000567
20                  M\PYGZus{}sails\PYGZus{}total\PYGZus{}COG.y  0.000005
21                  M\PYGZus{}sails\PYGZus{}total\PYGZus{}COG.z  0.000733
22           M\PYGZus{}sails\PYGZus{}total\PYGZus{}COW.x (heel) \PYGZhy{}0.000567
23          M\PYGZus{}sails\PYGZus{}total\PYGZus{}COW.y (pitch)  0.000005
24  M\PYGZus{}sails\PYGZus{}total\PYGZus{}COW.z (yaw \PYGZhy{} JG sign) \PYGZhy{}0.000733
25            M\PYGZus{}sails\PYGZus{}total\PYGZus{}COW.z (yaw)  0.000733
26                        M\PYGZus{}total\PYGZus{}COG.x \PYGZhy{}0.003889
27                        M\PYGZus{}total\PYGZus{}COG.y  0.000024
28                        M\PYGZus{}total\PYGZus{}COG.z  0.000733
29                 M\PYGZus{}total\PYGZus{}COW.x (heel) \PYGZhy{}0.003889
30                M\PYGZus{}total\PYGZus{}COW.y (pitch)  0.000024
31        M\PYGZus{}total\PYGZus{}COW.z (yaw \PYGZhy{} JG sign) \PYGZhy{}0.000733
32                  M\PYGZus{}total\PYGZus{}COW.z (yaw)  0.000733
\end{sphinxVerbatim}

\end{sphinxuseclass}\end{sphinxVerbatimOutput}

\end{sphinxuseclass}
\begin{sphinxuseclass}{cell}\begin{sphinxVerbatimInput}

\begin{sphinxuseclass}{cell_input}
\begin{sphinxVerbatim}[commandchars=\\\{\}]
\PYG{n}{AR} \PYG{o}{=} \PYG{l+m+mi}{2} \PYG{o}{*} \PYG{n}{main\PYGZus{}sail\PYGZus{}luff} \PYG{o}{/} \PYG{n}{main\PYGZus{}sail\PYGZus{}chords}\PYG{p}{[}\PYG{l+m+mi}{0}\PYG{p}{]}
\PYG{n}{S} \PYG{o}{=} \PYG{l+m+mi}{2}\PYG{o}{*} \PYG{n}{main\PYGZus{}sail\PYGZus{}luff} \PYG{o}{*} \PYG{n}{main\PYGZus{}sail\PYGZus{}chords}\PYG{p}{[}\PYG{l+m+mi}{0}\PYG{p}{]}

\PYG{n}{C\PYGZus{}xyz} \PYG{o}{=} \PYG{n}{get\PYGZus{}vlm\PYGZus{}Cxyz}\PYG{p}{(}\PYG{n}{myvlm}\PYG{o}{.}\PYG{n}{force}\PYG{p}{,} \PYG{n}{np}\PYG{o}{.}\PYG{n}{array}\PYG{p}{(}\PYG{n}{w\PYGZus{}profile}\PYG{o}{.}\PYG{n}{get\PYGZus{}true\PYGZus{}wind\PYGZus{}speed\PYGZus{}at\PYGZus{}h}\PYG{p}{(}\PYG{l+m+mf}{1.0}\PYG{p}{)}\PYG{p}{)}\PYG{p}{,} \PYG{n}{rho}\PYG{p}{,} \PYG{n}{S}\PYG{p}{)}
\PYG{n}{C\PYGZus{}xyz}
\end{sphinxVerbatim}

\end{sphinxuseclass}\end{sphinxVerbatimInput}
\begin{sphinxVerbatimOutput}

\begin{sphinxuseclass}{cell_output}
\begin{sphinxVerbatim}[commandchars=\\\{\}]
(0.0003111088732027318, 0.05423878234346646, 8.297465605756411e\PYGZhy{}20)
\end{sphinxVerbatim}

\end{sphinxuseclass}\end{sphinxVerbatimOutput}

\end{sphinxuseclass}
\begin{sphinxuseclass}{cell}\begin{sphinxVerbatimInput}

\begin{sphinxuseclass}{cell_input}
\begin{sphinxVerbatim}[commandchars=\\\{\}]
\PYG{c+c1}{\PYGZsh{} nie dziala na razie to tak jak zaplanowalam}
\PYG{c+c1}{\PYGZsh{} cel: by funcja wywołana z notebooka pokazywala na ekranie rysunek }
\PYG{c+c1}{\PYGZsh{} a wywolana ze skryptu zapisywala do pliku}
\PYG{c+c1}{\PYGZsh{} aktualnie w obu przypadkach bedzie zapisywac do pliku html, mozna}
\PYG{c+c1}{\PYGZsh{} go otworzyc interaktywnie w przegladarce}
\PYG{n}{plot\PYGZus{}cp}\PYG{p}{(}\PYG{n}{sail\PYGZus{}set}\PYG{o}{.}\PYG{n}{zero\PYGZus{}mesh}\PYG{p}{,} \PYG{n}{myvlm}\PYG{o}{.}\PYG{n}{p\PYGZus{}coeffs}\PYG{p}{,} \PYG{n}{output\PYGZus{}dir\PYGZus{}name}\PYG{p}{)}
\end{sphinxVerbatim}

\end{sphinxuseclass}\end{sphinxVerbatimInput}
\begin{sphinxVerbatimOutput}

\begin{sphinxuseclass}{cell_output}
\end{sphinxuseclass}\end{sphinxVerbatimOutput}

\end{sphinxuseclass}
\sphinxAtStartPar
There is a lot more that you can do with outputs (such as including interactive outputs)
with your book. For more information about this, see \sphinxhref{https://jupyterbook.org}{the Jupyter Book documentation}

\sphinxstepscope


\chapter{bibliography}
\label{\detokenize{chapters/bibliography:bibliography}}\label{\detokenize{chapters/bibliography::doc}}
\begin{sphinxthebibliography}{SHR+19}
\bibitem[AH13]{chapters/bibliography:id2}
\sphinxAtStartPar
T.Larsson A. Helmstad. \sphinxstyleemphasis{An Aeroelastic Implementation for Yacht Sails and Rigs, Degree project in Naval Architecture}. KTH Engeneering Sciences, 2013.
\bibitem[JK01]{chapters/bibliography:id3}
\sphinxAtStartPar
A. Plotkin J. Katz. \sphinxstyleemphasis{Low\sphinxhyphen{}Speed Aerodynamics, Second Edition}. Cambridge University Press, 2001.
\bibitem[JJB09]{chapters/bibliography:id4}
\sphinxAtStartPar
Russell M. Cummings John J. Bertin. \sphinxstyleemphasis{Aerodynamics for engineers, Fifth Edition}. Pearson Education International, 2009.
\bibitem[PS00]{chapters/bibliography:id6}
\sphinxAtStartPar
W. F. Phillips and D. O. Snyder. Modern adaptation of prandtl's classic lifting\sphinxhyphen{}line theory. \sphinxstyleemphasis{JOURNAL OF AIRCRAFT}, 37(4):662–670, July \sphinxhyphen{} August 2000. URL: \sphinxurl{http://arc.aiaa.org}, \sphinxhref{https://doi.org/DOI: 10.2514/2.2649}{doi:DOI: 10.2514/2.2649}.
\bibitem[SHR+19]{chapters/bibliography:id7}
\sphinxAtStartPar
A. Septiyana, K. Hidayat, A. Rizaldi, M. L. Ramadiansyah, R. A. Ramadhan, P. A. P. Suseno, E. B. Jayanti, N. Atmasari, and A. Rasyadi. Analysis of aerodynamic characteristics using the vortex lattice method on twin tail boom unmanned aircraft. \sphinxstyleemphasis{7TH INTERNATIONAL SEMINAR ON AEROSPACE SCIENCE AND TECHNOLOGY \sphinxhyphen{} ISAST 2019}, 2226(April):536, 24\sphinxhyphen{}25 September 2019. URL: \sphinxurl{https://pubs.aip.org/aip/acp/article/2226/1/020003/814546/Analysis-of-aerodynamic-characteristics-using-the}, \sphinxhref{https://doi.org/10.1063/5.0002337}{doi:10.1063/5.0002337}.
\bibitem[Tec05]{chapters/bibliography:id5}
\sphinxAtStartPar
Prof. A.H. Techet. Potential flow theory. In \sphinxstyleemphasis{Lecture: 2.016 Hydrodynamics}. Massachusetts Institute of Technology, 2005. MIT. URL: \sphinxurl{https://web.mit.edu/2.016/www/handouts/2005Reading4.pdf}.
\end{sphinxthebibliography}







\renewcommand{\indexname}{Index}
\printindex
\end{document}