%% Generated by Sphinx.
\def\sphinxdocclass{jupyterBook}
\documentclass[a4paper,12pt,english]{jupyterBook}
\ifdefined\pdfpxdimen
   \let\sphinxpxdimen\pdfpxdimen\else\newdimen\sphinxpxdimen
\fi \sphinxpxdimen=.75bp\relax
\ifdefined\pdfimageresolution
    \pdfimageresolution= \numexpr \dimexpr1in\relax/\sphinxpxdimen\relax
\fi
%% let collapsible pdf bookmarks panel have high depth per default
\PassOptionsToPackage{bookmarksdepth=5}{hyperref}
%% turn off hyperref patch of \index as sphinx.xdy xindy module takes care of
%% suitable \hyperpage mark-up, working around hyperref-xindy incompatibility
\PassOptionsToPackage{hyperindex=false}{hyperref}
%% memoir class requires extra handling
\makeatletter\@ifclassloaded{memoir}
{\ifdefined\memhyperindexfalse\memhyperindexfalse\fi}{}\makeatother

\PassOptionsToPackage{warn}{textcomp}

\catcode`^^^^00a0\active\protected\def^^^^00a0{\leavevmode\nobreak\ }
\usepackage{cmap}
\usepackage{fontspec}
\defaultfontfeatures[\rmfamily,\sffamily,\ttfamily]{}
\usepackage{amsmath,amssymb,amstext}
\usepackage{polyglossia}
\setmainlanguage{english}



\setmainfont{FreeSerif}[
  Extension      = .otf,
  UprightFont    = *,
  ItalicFont     = *Italic,
  BoldFont       = *Bold,
  BoldItalicFont = *BoldItalic
]
\setsansfont{FreeSans}[
  Extension      = .otf,
  UprightFont    = *,
  ItalicFont     = *Oblique,
  BoldFont       = *Bold,
  BoldItalicFont = *BoldOblique,
]
\setmonofont{FreeMono}[
  Extension      = .otf,
  UprightFont    = *,
  ItalicFont     = *Oblique,
  BoldFont       = *Bold,
  BoldItalicFont = *BoldOblique,
]



\usepackage[Bjarne]{fncychap}
\usepackage[,numfigreset=1,mathnumfig]{sphinx}

\fvset{fontsize=\small}
\usepackage{geometry}
\usepackage{nicefrac, xfrac}


% Include hyperref last.
\usepackage{hyperref}
% Fix anchor placement for figures with captions.
\usepackage{hypcap}% it must be loaded after hyperref.
% Set up styles of URL: it should be placed after hyperref.
\urlstyle{same}


\usepackage{sphinxmessages}



        % Start of preamble defined in sphinx-jupyterbook-latex %
         \usepackage[Latin,Greek]{ucharclasses}
        \usepackage{unicode-math}
        % fixing title of the toc
        \addto\captionsenglish{\renewcommand{\contentsname}{Contents}}
        \hypersetup{
            pdfencoding=auto,
            psdextra
        }
        % End of preamble defined in sphinx-jupyterbook-latex %
        

\title{Validation}
\date{Jun 09, 2023}
\release{}
\author{Zuzanna Wieczorek, Grzegorz Gruszczyński}
\newcommand{\sphinxlogo}{\vbox{}}
\renewcommand{\releasename}{}
\makeindex
\begin{document}

\pagestyle{empty}
\sphinxmaketitle
\clearpage

\pagestyle{plain}
\sphinxtableofcontents
\pagestyle{normal}
\phantomsection\label{\detokenize{chapters/validation/validation::doc}}


\sphinxAtStartPar
To validate result obtained from pySailingVLM two different wings were tested: swept Bertin wing and rectangular plat one.

\sphinxstepscope


\part{Rectangular flat plate}
\label{\detokenize{chapters/validation/flat_plate:rectangular-flat-plate}}\label{\detokenize{chapters/validation/flat_plate::doc}}
\sphinxAtStartPar
Using pySailingVLm a rectangular flat plate was modeled with a span of 10 units and a constant chord of 1 unit. The geomery of sail is discretized into 16 spanwise and 8 chordwise panels. The free\sphinxhyphen{}stream strength is 1 unit and the angle of attack is \(10^\circ\). Then the lift coefficient slope  \(C_{L,\alpha}\) was compared with {[}\hyperlink{cite.chapters/bibliography:id3}{AH13}{]} and analitically derived results in the table below.


\begin{savenotes}\sphinxattablestart
\centering
\sphinxcapstartof{table}
\sphinxthecaptionisattop
\sphinxcaption{Results from pySailingVLM code of the rectangular flat plate of a slope }\label{\detokenize{chapters/validation/flat_plate:comp}}
\sphinxaftertopcaption
\begin{tabulary}{\linewidth}[t]{|T|T|T|T|}
\hline
\sphinxstyletheadfamily &\sphinxstyletheadfamily 
\sphinxAtStartPar
Analitical
&\sphinxstyletheadfamily 
\sphinxAtStartPar
{[}\hyperlink{cite.chapters/bibliography:id3}{AH13}{]}
&\sphinxstyletheadfamily 
\sphinxAtStartPar
pySailingVLM
\\
\hline
\sphinxAtStartPar
\(C_{L,\alpha}\)
&
\sphinxAtStartPar
4.896
&
\sphinxAtStartPar
4.786
&
\sphinxAtStartPar
4.846
\\
\hline
\end{tabulary}
\par
\sphinxattableend\end{savenotes}

\begin{figure}[htbp]
\centering
\capstart

\noindent\sphinxincludegraphics[height=400\sphinxpxdimen]{{flat_cp}.png}
\caption{Pressure coefficients for flat plate. Figure generated by pySailingVLM.}\label{\detokenize{chapters/validation/flat_plate:flat-cp}}\end{figure}

\sphinxAtStartPar
????????????????????? dac cp plot z magisterki??????????????????????????????


\chapter{Results}
\label{\detokenize{chapters/validation/flat_plate:results}}
\sphinxAtStartPar
The results obtained by pySailingVLM are close to analitical and also {[}\hyperlink{cite.chapters/bibliography:id3}{AH13}{]} data. It should be noticed that analytically derived results assumes a span efficiency factor of 0.8.

\sphinxAtStartPar
????????????????????????? DAC wzor gdzies????????????????????

\sphinxstepscope


\part{Sweep wing}
\label{\detokenize{chapters/validation/sweep:sweep-wing}}\label{\detokenize{chapters/validation/sweep::doc}}
\sphinxAtStartPar
Swept wing by Bertin {[}\hyperlink{cite.chapters/bibliography:id5}{JJB09}{]}, is a flat plate with a \(45^\circ\) leading edge sweeping angle. The span of wing is 1 unit and a chord of 0.2 unit. Model was discretized into 4 spanwise panels and 1 chordwise panel. The free stream is aligned with the x\sphinxhyphen{}axis and has a magnitude of 1 units.


\begin{savenotes}\sphinxattablestart
\centering
\sphinxcapstartof{table}
\sphinxthecaptionisattop
\sphinxcaption{Comparison between swept wing by Bertin and pySailingVLM results.}\label{\detokenize{chapters/validation/sweep:comp-sweep}}
\sphinxaftertopcaption
\begin{tabulary}{\linewidth}[t]{|T|T|T|}
\hline
\sphinxstyletheadfamily &\sphinxstyletheadfamily 
\sphinxAtStartPar
Bertin
&\sphinxstyletheadfamily 
\sphinxAtStartPar
pySailingVLM
\\
\hline
\sphinxAtStartPar
\(C_{L,\alpha}\)
&
\sphinxAtStartPar
3.443
&
\sphinxAtStartPar
3.434
\\
\hline
\end{tabulary}
\par
\sphinxattableend\end{savenotes}

\begin{figure}[htbp]
\centering
\capstart

\noindent\sphinxincludegraphics{{bertin_1}.png}
\caption{Lift coefficient versus angle of attack. Figure generated by pySailingVLM.}\label{\detokenize{chapters/validation/sweep:bertin-1}}\end{figure}

\begin{figure}[htbp]
\centering
\capstart

\noindent\sphinxincludegraphics{{bertin_2}.png}
\caption{Comparison of spanwise lift distributionfor the Bertin wing. Figure generated by pySailingVLM.}\label{\detokenize{chapters/validation/sweep:bertin-2}}\end{figure}

\sphinxAtStartPar
????????????????????????
TODO
Dodatkowy wykres za Katzem: https://www.researchgate.net/publication/245355607\_Calculation\_of\_the\_Aerodynamic\_Forces\_on\_Automotive\_Lifting\_Surfaces
Fig4


\part{Results}
\label{\detokenize{chapters/validation/sweep:results}}
\sphinxAtStartPar
Results obtained by pySailingVLM are close to teoretical results taken from {[}\hyperlink{cite.chapters/bibliography:id5}{JJB09}{]}.
When the angle of attack increases, the lift coefficient calculated by pySailingVLM differ from Bertin results (figure \hyperref[\detokenize{chapters/validation/sweep:bertin-1}]{\ref{\detokenize{chapters/validation/sweep:bertin-1}}}). It can be related with a small\sphinxhyphen{}angle approximation used in the Vortex Lattice Method, where the angle of attack and the angle of sideslip should be both small.
??????????????????????????????????????????







\renewcommand{\indexname}{Index}
\printindex
\end{document}