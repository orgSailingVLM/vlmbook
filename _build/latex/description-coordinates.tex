%% Generated by Sphinx.
\def\sphinxdocclass{jupyterBook}
\documentclass[a4paper,12pt,english]{jupyterBook}
\ifdefined\pdfpxdimen
   \let\sphinxpxdimen\pdfpxdimen\else\newdimen\sphinxpxdimen
\fi \sphinxpxdimen=.75bp\relax
\ifdefined\pdfimageresolution
    \pdfimageresolution= \numexpr \dimexpr1in\relax/\sphinxpxdimen\relax
\fi
%% let collapsible pdf bookmarks panel have high depth per default
\PassOptionsToPackage{bookmarksdepth=5}{hyperref}
%% turn off hyperref patch of \index as sphinx.xdy xindy module takes care of
%% suitable \hyperpage mark-up, working around hyperref-xindy incompatibility
\PassOptionsToPackage{hyperindex=false}{hyperref}
%% memoir class requires extra handling
\makeatletter\@ifclassloaded{memoir}
{\ifdefined\memhyperindexfalse\memhyperindexfalse\fi}{}\makeatother

\PassOptionsToPackage{warn}{textcomp}

\catcode`^^^^00a0\active\protected\def^^^^00a0{\leavevmode\nobreak\ }
\usepackage{cmap}
\usepackage{fontspec}
\defaultfontfeatures[\rmfamily,\sffamily,\ttfamily]{}
\usepackage{amsmath,amssymb,amstext}
\usepackage{polyglossia}
\setmainlanguage{english}



\setmainfont{FreeSerif}[
  Extension      = .otf,
  UprightFont    = *,
  ItalicFont     = *Italic,
  BoldFont       = *Bold,
  BoldItalicFont = *BoldItalic
]
\setsansfont{FreeSans}[
  Extension      = .otf,
  UprightFont    = *,
  ItalicFont     = *Oblique,
  BoldFont       = *Bold,
  BoldItalicFont = *BoldOblique,
]
\setmonofont{FreeMono}[
  Extension      = .otf,
  UprightFont    = *,
  ItalicFont     = *Oblique,
  BoldFont       = *Bold,
  BoldItalicFont = *BoldOblique,
]



\usepackage[Bjarne]{fncychap}
\usepackage[,numfigreset=1,mathnumfig]{sphinx}

\fvset{fontsize=\small}
\usepackage{geometry}
\usepackage{nicefrac, xfrac}


% Include hyperref last.
\usepackage{hyperref}
% Fix anchor placement for figures with captions.
\usepackage{hypcap}% it must be loaded after hyperref.
% Set up styles of URL: it should be placed after hyperref.
\urlstyle{same}


\usepackage{sphinxmessages}



        % Start of preamble defined in sphinx-jupyterbook-latex %
         \usepackage[Latin,Greek]{ucharclasses}
        \usepackage{unicode-math}
        % fixing title of the toc
        \addto\captionsenglish{\renewcommand{\contentsname}{Contents}}
        \hypersetup{
            pdfencoding=auto,
            psdextra
        }
        % End of preamble defined in sphinx-jupyterbook-latex %
        

\title{Coordinate System and Wind}
\date{Jun 09, 2023}
\release{}
\author{Zuzanna Wieczorek, Grzegorz Gruszczyński}
\newcommand{\sphinxlogo}{\vbox{}}
\renewcommand{\releasename}{}
\makeindex
\begin{document}

\pagestyle{empty}
\sphinxmaketitle
\clearpage

\pagestyle{plain}
\sphinxtableofcontents
\pagestyle{normal}
\phantomsection\label{\detokenize{chapters/description/coordinates::doc}}


\sphinxAtStartPar
The surface of the geometry used is defined with coordinates in three dimensions (figure \hyperref[\detokenize{chapters/description/coordinates:geoms}]{\ref{\detokenize{chapters/description/coordinates:geoms}}})  with the x\sphinxhyphen{}axis in the yachts longitudinal (chordwise) direction positive backwards, the y\sphinxhyphen{}axis in the transverse direction positive to starboard and the z\sphinxhyphen{}axis in spanwise direction positive upwards.

\begin{figure}[htbp]
\centering
\capstart

\noindent\sphinxincludegraphics[height=500\sphinxpxdimen]{{geom}.png}
\caption{Sail coordinate system. Figure by author.}\label{\detokenize{chapters/description/coordinates:geoms}}\end{figure}

\sphinxAtStartPar
A boat moves along x\sphinxhyphen{}axis with the \(\overrightarrow{V}_{yacht}\) velocity (figure \hyperref[\detokenize{chapters/description/coordinates:coord-sys}]{\ref{\detokenize{chapters/description/coordinates:coord-sys}}}). The speed of yacht  measured directly from the sailor point of view is desscribed by vectors \(\overrightarrow{V}_{app\_wind\_infs}\) (without induced wind velocity \(\overrightarrow{w}\) ) and \(\overrightarrow{V}_{app\_wind_\fs}\) (with induced wind velocity). An aerodynamic force is generated and is  perpendicular to the apparent wind velocity. The profile of apparent wind is twisted (see figure \hyperref[\detokenize{chapters/description/coordinates:north}]{\ref{\detokenize{chapters/description/coordinates:north}}}).

\begin{figure}[htbp]
\centering
\capstart

\noindent\sphinxincludegraphics[height=500\sphinxpxdimen]{{coord_sys}.png}
\caption{Yacht coordinate system. Figure  from {[}\hyperlink{cite.chapters/bibliography:id7}{Gru21}{]} p.18.}\label{\detokenize{chapters/description/coordinates:coord-sys}}\end{figure}


\begin{savenotes}\sphinxattablestart
\centering
\sphinxcapstartof{table}
\sphinxthecaptionisattop
\sphinxcaption{Nomenclature}\label{\detokenize{chapters/description/coordinates:coord-table}}
\sphinxaftertopcaption
\begin{tabulary}{\linewidth}[t]{|T|T|}
\hline
\sphinxstyletheadfamily 
\sphinxAtStartPar
Name
&\sphinxstyletheadfamily 
\sphinxAtStartPar
Description
\\
\hline
\sphinxAtStartPar
\(\overrightarrow{V}_{yacht}\)
&
\sphinxAtStartPar
angle between apparent wind and true wind
\\
\hline
\sphinxAtStartPar
\(\overrightarrow{V}_{app\_wind\_infs\_k}\)
&
\sphinxAtStartPar
apparent wind velocity for an ‘infinite sail’ (without induced wind velocity) acting on k\sphinxhyphen{}th panel
\\
\hline
\sphinxAtStartPar
\(\overrightarrow{V}_{app\_wind\_fs\_k}\)
&
\sphinxAtStartPar
apparent wind velocity for an ‘finite sail’ (with induced wind velocity) acting on k\sphinxhyphen{}th panel
\\
\hline
\sphinxAtStartPar
\(\overrightarrow{V}\_{true\_wind\_k}\)
&
\sphinxAtStartPar
true wind velocity acting on k\sphinxhyphen{}th panel
\\
\hline
\sphinxAtStartPar
\(\overrightarrow{w}\_{k}\)
&
\sphinxAtStartPar
induced wind velocity acting on k\sphinxhyphen{}th panel
\\
\hline
\sphinxAtStartPar
\(\alpha_{app\_infs\_k}\)
&
\sphinxAtStartPar
angle between apparent wind acting on k\sphinxhyphen{}th panel for an ‘infinite sail’ (without induced wind velocity) and direction of boat movement (including leeway)
\\
\hline
\sphinxAtStartPar
\(\alpha_{app\_fs\_k}\)
&
\sphinxAtStartPar
angle between apparent wind of a ‘finite sail’ acting on k\sphinxhyphen{}th panel (with induced wind velocity) and direction of boat movement (including leeway)
\\
\hline
\sphinxAtStartPar
\(\alpha_{ind\_k}\)
&
\sphinxAtStartPar
angle between apparent wind acting on k\sphinxhyphen{}th panel of a ‘finite sail’ (induced wind velocity) and apparent wind of an ‘infinite sail’ (without induced wind velocity)
\\
\hline
\sphinxAtStartPar
\(\alpha_{yacht\_k}\)
&
\sphinxAtStartPar
angle between apparent wind acting on k\sphinxhyphen{}th panel and true wind
\\
\hline
\sphinxAtStartPar
\(\alpha_{true}\)
&
\sphinxAtStartPar
angle between true wind acting on k\sphinxhyphen{}th panel and direction of boat movement with reference to course over ground (i.e. including leeway) {[}deg{]}
\\
\hline
\sphinxAtStartPar
\(\overrightarrow{F}_a\)
&
\sphinxAtStartPar
aerodynamic force acting on k\sphinxhyphen{}th panel
\\
\hline
\sphinxAtStartPar
\(\overrightarrow{T}\)
&
\sphinxAtStartPar
thrust force in direction of boat movement \sphinxhyphen{} Speed Over Ground (SOG)
\\
\hline
\end{tabulary}
\par
\sphinxattableend\end{savenotes}

\begin{figure}[htbp]
\centering
\capstart

\noindent\sphinxincludegraphics[height=300\sphinxpxdimen]{{North-Sail-Understanding-Twised_wind}.png}
\caption{Profile of apparent wind on sails {[}\hyperlink{cite.chapters/bibliography:id2}{NS18}{]}.}\label{\detokenize{chapters/description/coordinates:north}}\end{figure}

\sphinxAtStartPar
It is worth adding that the induced wind causes extra drag (see figure \hyperref[\detokenize{chapters/description/coordinates:induced-drag}]{\ref{\detokenize{chapters/description/coordinates:induced-drag}}})

\begin{figure}[htbp]
\centering
\capstart

\noindent\sphinxincludegraphics[height=300\sphinxpxdimen]{{induced_drag}.png}
\caption{Extra drag generated by induced wind. Figure 6 from {[}\hyperlink{cite.chapters/bibliography:id7}{Gru21}{]}.}\label{\detokenize{chapters/description/coordinates:induced-drag}}\end{figure}







\renewcommand{\indexname}{Index}
\printindex
\end{document}